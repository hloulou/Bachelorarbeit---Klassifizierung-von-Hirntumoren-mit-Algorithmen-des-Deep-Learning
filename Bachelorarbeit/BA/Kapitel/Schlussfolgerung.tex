\section{Schlussfolgerung}

Im Rahmen dieser Arbeit wurde die Klassifikation von \ac{ht} unter Zuhilfenahme von \ac{dl}-Algorithmen untersucht. Zu diesem Zweck wurde ein eigener \ac{cnn}-Ansatz mit 7.023 medizinischen Bildern trainiert. Die Herausforderung, ohne Datenaugmentation zu arbeiten, wurde durch die realistische Nähe zu klinischen Daten motiviert, was insbesondere im Hinblick auf das Risiko des Overfitting von Bedeutung war. Durch den Einsatz von Regularisierungstechniken wie Dropout, Batch-Normalisierung und L2-Regularisierung konnte eine Reduktion des Overfittings erzielt werden. Obwohl der Datensatz lediglich 7.023 Bilder umfasst, konnte das Modell ein Testgenauigkeit von 98,1\% sowie einen Testverlust von 0,1027 erzielen, was als gut zu bewerten ist.
\\
Zudem wurde ein Vergleich des eigenentwickelten Modells mit dem vortrainierten ResNet-50-Modell vorgenommen, um die Vorzüge von \ac{tl} und von \ac{ft} zu nutzen. Die ursprüngliche Architektur des ResNet-50 wurde beibehalten, wobei die Ausgabeschicht angepasst wurde. Das Modell wurde mit einer definierten Anzahl an trainierbaren Schichten an dem Datensatz feingetunt. Die Durchführung von \ac{hpo} ermöglichte die Untersuchung des Einflusses der Anzahl trainierbarer Schichten auf die Modellleistung. Die Ergebnisse zeigten, dass die Konsistenz der Resultate ab einer Anzahl von 19 Schichten gegeben war. Das optimierte vortrainierte Modell erreichte eine Testgenauigkeit von 97,66\% bei einem Testverlust von 0,1097. 
\\
Die ursprüngliche Hypothese, dass das vortrainierte Modell zu besseren Resultaten führen würde, konnte in dieser Untersuchung nicht bestätigt werden. Das vortrainierte Modell erzielte ähnliche Resultate zum selbst entwickelten Modell, wies jedoch kein Overfitting auf, wodurch es sich unter bestimmten Voraussetzungen für die Klassifikation von \ac{ht} als potenziell besser geeignet erwies. Für Experten im Bereich des \ac{ml}s und des \ac{dl}s könnte das selbst entwickelte Modell aufgrund der spezifischen Anpassung an den Datensatz sowie der Möglichkeit einer erweiterten \ac{hpo} von besonderem Vorteil sein. In Bezug auf die Anwendbarkeit in klinischen Umgebungen, insbesondere zur Unterstützung von Radiologen, stellt das vortrainierte Modell ResNet-50 aufgrund seiner einfachen Implementierung und der guten Ergebnisse eine attraktive Wahl dar.
\\
In künftigen Forschungsarbeiten sollte untersucht werden, inwiefern eine Erweiterung des Datensatzes durch synthetische Daten oder zusätzliche klinische Datensätze eine weitere Verbesserung der Performance beider Modelle ermöglicht. Des Weiteren erscheint eine Vertiefung der \ac{hpo} sowie eine Prüfung des Einflusses zusätzlicher Regularisierungstechniken sinnvoll, um das Modell weiter zu optimieren und eine Robustheit gegen Overfitting zu erzielen. 
\\
Zusammenfassend lässt sich sagen, dass \ac{dl} ein vielversprechendes Werkzeug zur Unterstützung der medizinischen Diagnostik darstellt, dessen Potenzial durch kontinuierliche Forschung und Optimierung noch weiter ausgeschöpft werden kann.