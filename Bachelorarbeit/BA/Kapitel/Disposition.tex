\section*{Disposition}
\addcontentsline{toc}{section}{Disposition}
% add the unnumbered section to the table of contents
%\begin{abstract}
Im Rahmen dieser Arbeit erfolgt eine Untersuchung des Themas \textbf{"`Klassifizierung von Hirntumoren mit Algorithmen des Deep Learning"'}. Das Ziel dieser Untersuchung ist die Analyse der Performance von \textbf{\acf{cnn}} bei der Klassifizierung von \acf{ht} anhand von \acf{mrt}-Bildern.\\
Die zentrale Fragestellung lautet: Welcher Unterschied besteht in der Performance zwischen einem selbst entwickelten \ac{cnn}-Modell und einem vortrainierten Modell bei der Klassifizierung von \acs{ht}. Dabei wird evaluiert, ob ein durch \textbf{\acf{ft}} an die \ac{mrt}-Daten optimiertes vortrainiertes \ac{cnn}-Modell eine höhere Klassifikationsgenauigkeit erzielt als ein neu entwickeltes und vollständig trainiertes \ac{cnn}. Im Rahmen des \ac{ft} erfolgt die Feinabstimmung eines vortrainierten Modells auf einen spezifischen Datensatz durch Anpassung tiefer Schichten.\\
Die Hypothese besagt, dass \textbf{\acf{tl}}, bei dem ein Modell, das zuvor auf einem umfangreichen Datensatz wie \textbf{ImageNet} trainiert wurde, in Kombination mit \ac{ft} eine signifikante Verbesserung der Klassifikationsleistung bewirkt.\\
Zur Beantwortung der Fragestellung wird der öffentlich zugängliche \ac{ds} \citet[]{Nickparvar:2021} verwendet, der \textbf{7.023 \ac{mrt}-Bilder} verschiedener \ac{ht} sowie nicht tumoröser Gewebe enthält. Die Methodik beinhaltet sowohl die Implementierung eines eigenen \ac{cnn} als auch die Anpassung des vortrainierten ResNet-50-Modells von \citet[]{He:2016} an dem Datensatz. Die Modelle werden anhand von Metriken wie \textbf{Accuracy}, \textbf{Recall}, \textbf{Preciscion} und \textbf{F1-Score} evaluiert, ergänzt durch \textbf{Konfusionsmatrizen}. Die Implementierung wird in \textbf{Python} unter Verwendung von Bibliotheken wie \textbf{TensorFlow} und \textbf{Keras} durchgeführt.\\
Die Arbeit ist in mehrere Abschnitte gegliedert. Nach der Einführung in das Thema sowie einem Literaturüberblick erfolgt die Erläuterung des theoretischen Hintergrunds der Modelle und der Evaluierungsmethoden. Der methodische Teil beschreibt den Datensatz und die Modellierungsansätze begleitet von Architekturdiagrammen. Daraufhin werden die Resultate präsentiert und analysiert sowie im Kontext der formulierten Forschungsfragen und Hypothesen diskutiert. Abschließend werden die wesentlichen Erkenntnisse zusammengefasst und künftige Forschungsperspektiven in diesem Bereich erörtert.\\
Das Literaturverzeichnis enthält aktuelle, relevante und zentrale wissenschaftliche Artikel und Bücher, die als theoretische und methodische Grundlage dienen.
%\end{abstract}