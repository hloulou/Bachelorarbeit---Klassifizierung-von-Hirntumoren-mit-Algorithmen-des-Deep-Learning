\section{Zukunftsaussichten}

Die Weiterentwicklung der Klassifizierung von \ac{ht} mithilfe von \ac{dl}-Algorithmen eröffnet zahlreiche Perspektiven zur Leistungssteigerung und Bewältigung bestehender Herausforderungen. Ein wesentliches Problem stellt das Overfitting dar, welches durch eine Erweiterung des Datensatzes reduziert werden könnte. Eine Lösung könnte in der Kombination verschiedener öffentlich zugänglicher \ac{mrt}-Datensätze oder der Ergänzung um \ac{ct}-Bilder bestehen. Die üblicherweise begrenzte Größe medizinischer Bilddatensätze sowie deren hohe Spezialisierung könnten durch einen umfassenderen Datensatz kompensiert werden. Dies würde die Robustheit und Generalisierungsfähigkeit der Modelle fördern und zu stabileren Trainingsergebnissen sowie einer besseren Modellvalidität und -anpassungsfähigkeit gegenüber neuen, unbekannten Bildmustern führen. \\
Ein weiterer wesentlicher Ansatzpunkt zur Verbesserung der Modellleistung stellt die Optimierung der \ac{hp} dar. Ein fortschrittliches Verfahren in diesem Bereich ist das Iterated Racing von \citet{Birattari:2010}, welches von \citet[]{Bischl:2023} als eine der führenden Methoden der \ac{hpo} hervorgehoben wird. Das Verfahren fokussiert die Rechenleistung auf vielversprechende Konfigurationen, indem weniger geeignete Parameterkombinationen bereits in frühen Phasen eliminiert werden. Dies erlaubt eine effizientere und präzisere Suche nach optimalen Parametereinstellungen bei gleichzeitiger Schonung von Rechenressourcen. Diese Methode kann dazu beitragen, künftige Optimierungsprozesse zu beschleunigen und die Genauigkeit der Klassifikationsmodelle weiter zu steigern. \\
Eine weiterer innovativer Ansatz zur Steigerung der Leistungsfähigkeit stellt der Einsatz vortrainierter Modelle dar, welche auf verschiedene Tumorarten spezialisiert sind. Eine derartige Vortrainierung ermöglicht es dem Modell, allgemeine Tumorcharakteristika besser zu erkennen, wodurch sich zugleich seine Fähigkeit erhöht, spezifische \ac{ht}-Merkmale zu klassifizieren. Auch die Implementierung eines Segmentierungsansatzes birgt ein vielversprechendes Potenzial. Hierbei könnte das Modell gezielt lernen, Tumorstrukturen in Bildern zu identifizieren und anschließend zu klassifizieren. Der zusätzliche Vorverarbeitungsschritt könnte zu differenzierteren Einblicken in die Tumorbiologie führen, was insbesondere für die präzisere Erkennung seltener oder atypischer \ac{ht} von Bedeutung wäre. \\
Im Rahmen künftiger Forschungsarbeiten wäre es vorteilhaft, die Eigenschaften unterschiedlicher Modellarchitekturen systematisch zu analysieren und deren Leistungsfähigkeit im Hinblick auf medizinische Bilddaten zu evaluieren. Eine detaillierte Bewertung der Merkmalsextraktionsfähigkeiten könnte dazu beitragen, die am besten geeignete Architektur für die Klassifikation von \ac{ht} zu identifizieren und somit die Leistungsfähigkeit weiter zu steigern. Derartige Vergleiche könnten zudem Aufschluss über spezifische Architekturanpassungen geben, die sich besonders für medizinische Bildanalysen eignen.
\\
Ein vielversprechender Forschungsbereich ist zudem die Erweiterung der Modelle auf 3D-\ac{mrt}-Daten, um volumetrische Informationen besser auszunutzen. Dies könnte durch die Integration von 3D-\ac{cnn}s oder hybridisierten Modellen erfolgen, welche sowohl 2D- als auch 3D-Daten kombinieren. Darüber hinaus besteht ein Bedarf an der Entwicklung von Modellen, welche interpretierbare und klinisch relevante Ergebnisse liefern, um die Akzeptanz durch Mediziner zu fördern.
\\
Die Umsetzung der dargestellten Optimierungsansätze konnte im Rahmen dieser Arbeit jedoch nicht realisiert werden, da sie einen signifikanten Aufwand an Zeit und Rechenkapazität erfordert. Umfassende Datenerweiterungen, fortgeschrittene Transfer- oder Segmentierungsansätze sowie eine Hyperparameteroptimierung mittels Iterated Racing würden eine erhebliche Steigerung der verfügbaren Rechenleistung sowie einen erweiterten Zeitrahmen verlangen. Zukünftige Forschungsprojekte könnten daher gezielt auf diesen Ansätzen aufbauen und damit das Potenzial moderner \ac{dl}-Methoden zur \ac{ht}-Klassifikation weiter ausschöpfen.