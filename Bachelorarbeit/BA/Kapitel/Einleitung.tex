\section{Einführung}

In den vergangenen Jahren hat die rapide Entwicklung des \acf{ml} auf globaler Ebene zu einem Paradigmenwechsel geführt. Als besonders bemerkenswert erweist sich die Anwendung von \ac{ml} in der medizinischen Bildgebung, welche eine präzise und effiziente Diagnostik ermöglicht. Algorithmen des \acf{dl}, insbesondere \acf{cnn}, haben sich dabei als zentrale Technologie etabliert, da sie in der Lage sind, komplexe Muster in Bilddaten zu erkennen und zu klassifizieren. Diese Fortschritte eröffnen neue Perspektiven für die Diagnostik von Krankheiten, die eine schnelle und zuverlässige Analyse erfordern. (Vgl. \citealt[Vorwort]{Deprez:2023}). 
\\
\acf{ht} stellen ein erhebliches medizinisches Herausforderungsgebiet dar. Weltweit wird ihre Inzidenzrate mit 3,5 und die Mortalitätsrate mit 2,6 pro 100.000 Menschen angegeben (vgl. \citealt[]{Ferlay:2024}). Eine präzise Differenzierung zwischen den häufigsten Arten wie Gliomen (engl. \textit{glioma}), Meningeomen (engl. \textit{meningioma}) und Hypophysentumoren (engl. \textit{pituitary}), ist von grundlegender Bedeutung für die Auswahl einer geeigneten Therapie (vgl. \citealt[]{Bilsky:2023_}). Die Abbildung \ref{fig:1} veranschaulicht eine Auswahl von \ac{mrt}-Aufnahmen der drei \ac{ht}-Arten sowie eine \ac{mrt}-Aufnahme nicht tumorösen Gewebes (engl. \textit{no tumor}):
\begin{figure}[ht]
    \centering
        \includegraphics[width=0.65\textwidth]{BA/Abbildungen/Aufnahmearten.jpg}
        \caption{Auswahl von \ac{mrt}-Bildern der vier Aufnahmearten.}
    \label{fig:1}
\end{figure}\\
Die Diagnose von Hirntumoren basiert traditionell auf manuellen Analysen von Aufnahmen der \acf{mrt} und \acf{ct} durch Radiologen (vgl. \citealt[]{Oezkaraca:2023}). Der manuellen Überprüfung medizinischer Bilder ist jedoch ein hoher Zeitaufwand verbunden, zudem ist sie aufgrund des Patientenflusses potenziell fehleranfällig. Um dieses Problem zu lösen, ist die Entwicklung und Anwendung eines automatischen Bildanalyseverfahrens erforderlich, welches die Arbeitsbelastung bei der Klassifizierung und Diagnose von \ac{mrt}-Bildern des Gehirns reduziert, als Hilfsmittel für Radiologen und Ärzte dient und die Mortalität durch die Früherkennung von Tumoren senken kann (vgl. \citealt[Vorwort]{Deprez:2023}).
\\
In diesem Kontext hat sich \ac{dl} als vielversprechende Technologie erwiesen. \ac{cnn}s, die bereits bei Bildklassifikationsaufgaben wie der \ac{ilsvrc} \citet{Krizhevsky:2012} signifikante Erfolge erzielt haben, bieten auch im medizinischen Bereich ein beträchtliches Potenzial (vgl. \citealt[S. 103]{Lundervold:2019}). Die Fähigkeit zur automatischen Merkmalsextraktion stellt einen zentralen Vorteil von \ac{cnn} dar. Im Gegensatz zu klassischen Methoden erfordert ihre Architektur keine manuelle Feature-Definition, sondern extrahiert komplexe Merkmale direkt aus den Daten (vgl. \citealt[S. 105]{Lundervold:2019}). Dadurch sind sie für Aufgaben der medizinischen Bildverarbeitung geeignet, bei denen relevante Muster oft subtil und mehrdimensional sind.
\\
Der Einsatz von \ac{cnn}s ist jedoch mit Herausforderungen verbunden. Sie weisen einen hohen Rechenaufwand auf und sind insbesondere bei kleineren Datensätzen, die in der medizinischen Domäne häufig auftreten, anfällig für eine zu starke Anpassung an die Trainingsdaten. Dies kann zu einer eingeschränkten Generalisierungsfähigkeit auf neue Daten führen. (Vgl. \citealt[S. 28; 456]{Geron:2019}). 
\\
Vortrainierte \ac{cnn}-Modelle, die im Rahmen des \acf{tl} verwendet werden, stellen eine flexible Alternative dar. Sie ermöglichen die Übertragung bereits erlernter Merkmale aus großen Bilddatensätzen wie ImageNet \citet{Deng:2009} auf spezifische medizinische Aufgaben, wodurch der Bedarf an großen Datensätzen und Trainingsressourcen reduziert wird. Obwohl dies den Trainingsaufwand reduziert, könnte die Anpassungsfähigkeit solcher Modelle auf domänenspezifische Daten eingeschränkt sein. (Vgl. \citealt[S. 345-347]{Geron:2019}).
\\
Die vorliegende Arbeit befasst sich mit der Untersuchung des Einsatzes von \ac{cnn}s zur Klassifikation von \ac{ht} auf \ac{mrt}-Bildern. In diesem Kontext erfolgt eine Evaluierung eines selbst entwickelten, optimierten Modells sowie des vortrainierten Modells ResNet-50 von \citet{He:2016}. Ziel ist der Vergleich und die Analyse ihrer Performances anhand von verschiedenen Metriken. 
\\