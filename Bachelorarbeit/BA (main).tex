\documentclass[a4paper, 	        % Seitenformat
		12pt, 					% Schriftgröße
		bibliography=totoc, 	% Literaturverzeichnis in das Inhaltsverzeichnis
		index=totoc, 	        % Index in das Inhaltsverzeichnis
		abstract=true, 	        % mit Abstrakt
		headsepline, 	        % Trennlinie für die Kopfzeile
		%footnosepline,	        % Trennlinie für die Fusszeile
		]{scrartcl}

\usepackage[ngerman]{babel}	  % Mehrsprachenumgebung
\usepackage{latexsym}		      % Latex-Symbole
\usepackage[utf8]{inputenc}	% Eingabekodierungen
\usepackage{subfig}	  	% Unterteilungen von Abbildungen
\usepackage{float}		  % Gleitobjekte
\usepackage{graphicx}		% Einbinden von Graphiken aus Dateien
\usepackage{rotating}   % Die folgenden Packages sind meist hilfreich für das Erstellen von Arbeiten
\usepackage{mathrsfs}
\usepackage{eurosym,bm,amsmath,amssymb,verbatim}
\usepackage{listings}
\usepackage[T1]{fontenc}
\usepackage{setspace}
\usepackage [round,authoryear] {natbib} % Notwendiges Package für den verwendeten Literaturstil
\usepackage[printonlyused]{acronym}
\usepackage{hyperref}
\usepackage{url}
\usepackage{tabularx}
\usepackage{tikz}
\usepackage{bm}       % Für fette mathematische Symbole (z.B. Vektoren)
\usepackage{pgfplots}
\usepackage{caption} % Optional, für bessere Kontrolle über die Bildunterschriften
\usepackage{multirow}
\usepackage{adjustbox} % Optional für das Anpassen der Größen
\usepackage[dvipsnames]{xcolor} %Farbpalette
\usepackage{amsmath}
\usepackage{lscape} %dreht die Seite
\usepackage{makecell} % Tabellenzellen umbrechen
\usepackage{ragged2e} % Blocksatz 


%------------------------------------------------------------------------

\parindent0pt 			% Einzug nach einer Leerzeile
\topmargin-20.4mm   % Für die Ränder
\textheight245mm

\textwidth157mm \oddsidemargin11mm \evensidemargin11mm

% Nummerierung der Gleichungen nach Abschnitt
\renewcommand{\theequation}{\thesection.\arabic{equation}}
\numberwithin{equation}{section}

% Nummerierung der Abbildungen nach Abschnitt
\renewcommand{\thefigure}{\thesection.\arabic{figure}}
\numberwithin{figure}{section}

% Nummerierung der Tabellen nach Abschnitt
\renewcommand{\thetable}{\thesection.\arabic{table}}
\numberwithin{table}{section}

% Hyperlink-Optionen
\hypersetup{
    colorlinks=true,
    linkcolor=black,     
    urlcolor=cyan,
    citecolor=blue,
    pdfpagemode=FullScreen,
}

\pgfplotsset{compat=1.18}

% Definiere eine neue Liste für das Gleichungsverzeichnis
%\newcommand{\listequationsname}{Gleichungsverzeichnis}
%\newlistof{myequations}{equ}{\listequationsname}

% Befehl zur Aufnahme einer Gleichung ins Verzeichnis
%\newcommand{\myeqref}[1]{%
    %\refstepcounter{equ}%
    %\addcontentsline{equ}{myequations}{\protect\numberline{\theequ} (#1)}\par}

%%%%%%%%%%%%%%%%%%%%%%%%%%%%%%%%%%%%%%%%%%%%%%%%%%%%%%%%%%%%%%%%%%%%%%%
%%%%%%%%%%%%%%%%%%%%%%%%%%%%%%%%%%%%%%%%%%%%%%%%%%%%%%%%%%%%%%%%%%%%%%%
%%%%%%%%%%%%%%%%%%%%%%%%%%%%%%%%%%%%%%%%%%%%%%%%%%%%%%%%%%%%%%%%%%%%%%%

\begin{document}					
% Start des Dokuments

\begin{titlepage}

\vspace*{4.5cm}
\large\textbf{Klassifizierung von Hirntumoren mit Algorithmen des \\ Deep Learning} \\[0.5cm]


\normalsize Freie wissenschaftliche Arbeit\\

zur Erlangung des akademischen Grades\\

"`Bachelor of Science"' \\

Studiengang: Betriebswirtschaftslehre \\

\textbf{an der Wirtschaftswissenschaftlichen\\ Fakultät der Universität Augsburg}\\[0.8cm]

Lehrstuhl für Statistik\\[1.25cm]
%
%
\noindent\begin{tabular}[h]{@{}ll}

Eingereicht bei: & Prof. Dr. Yarema Okhrin\\
\\
Betreuerin: 	 & MSc. Sara Garber\\
\\
Vorgelegt von:   & Hanan Loulou\\
\\
\end{tabular}

\vfill
Augsburg, im Dezember 2024

\end{titlepage}
\newpage

%%%%%%%%%%%%%%%%%%%%%%%%%%%%%%%%%%%%%%%%%%%%%%%%%%%%%%%%%%%%%%%%%%%%%%%
%%%%%%%%%%%%%%%%%%%%%%%%%%%%%%%%%%%%%%%%%%%%%%%%%%%%%%%%%%%%%%%%%%%%%%%
%%%%%%%%%%%%%%%%%%%%%%%%%%%%%%%%%%%%%%%%%%%%%%%%%%%%%%%%%%%%%%%%%%%%%%%

\pagenumbering{roman}


\tableofcontents		    % Inhaltsverzeichnis
\newpage
\listoffigures			    % Abbildungsverzeichnis
% Fügt das Abbildungsverzeichnis ins Inhaltsverzeichnis ein
\addcontentsline{toc}{section}{Abbildungsverzeichnis}
\newpage
\listoftables				  % Tabellenverzeichnis
\addcontentsline{toc}{section}{Tabellenverzeichnis}
\newpage
%\listofmyequations          % Gleichungsverzeichnis 
%\newpage
\section*{Abkürzungsverzeichnis}
\begin{spacing}{0.2}
\begin{acronym}
\acro{abc}[ABC]{Artificial Bee Colony}
\acro{cce}[CCE]{Categorical Cross Entropy}
\acro{cr}[CR]{Classification Report}
\acro{cm}[CM]{Confusion Matrix}
\acro{cnn}[CNN]{Convolutional Neural Network}
\acro{ct}[CT]{Computertomographie}
\acro{dl}[DL]{Deep Learning}
\acro{dnn}[DNN]{Deep Neural Network}
\acro{dof}[DoF]{degrees of freedom}
\acro{ds}[DS]{Datensatz}
\acro{fn}[FN]{False Negative}
\acro{fnr}[FNR]{False Negative Rate}
\acro{fp}[FP]{False Positive}
\acro{fpa}[fpa]{Flower Pollination Algorithm}
\acro{fpr}[FPR]{False Positive Rate}
\acro{ft}[FT]{Fine-Tuning}
\acro{ga}[GA]{Gentischer Algorithmus}
\acro{gd}[GD]{Gradient Descent}
\acro{gs}[GS]{Grid Search}
\acro{ht}[HT]{Hirntumoren}
\acro{hp}[HP]{Hyperparameter}
\acro{hpo}[HPO]{Hyperparameteroptimierung}
\acro{ilsvrc}[ILSVRC]{ImageNet Large Scale Visual Recognition Challenge}
\acro{ml}[ML]{Machine Learning}
\acro{mlp}[MLP]{Multi-Layer-Perzeptron}
\acro{mrt}[MRT]{Magnetresonanztomographie}
\acro{mse}[MSE]{Mean Squared Error}
\acro{pso}[PSO]{Particle Swarm Optimization}
\acro{rs}[RS]{Random Search}
\acro{svm}[SVM]{Support Vector Machine}
\acro{tl}[TL]{Transfer Learning}
\acro{tn}[TN]{True Negative}
\acro{tnr}[TNR]{True Negative Rate}
\acro{tp}[TP]{True Positive}
\acro{tpr}[TPR]{True Positive Rate}
\end{acronym}
\end{spacing}

 % Abkürzungsverzeichnis einbinden 
\addcontentsline{toc}{section}{Abkürzungsverzeichnis}
%\printglossary[type=\acronymtype]
\newpage


%%%%%%%%%%%%%%%%%%%%%%%%%%%%%%%%%%%%%%%%%%%%%%%%%%%%%%%%%%%%%%%%%%%%%%%
%%%%%%%%%%%%%%%%%%%%%%%%%%%%%%%%%%%%%%%%%%%%%%%%%%%%%%%%%%%%%%%%%%%%%%%
%%%%%%%%%%%%%%%%%%%%%%%%%%%%%%%%%%%%%%%%%%%%%%%%%%%%%%%%%%%%%%%%%%%%%%%


\section*{Disposition}
\addcontentsline{toc}{section}{Disposition}
% add the unnumbered section to the table of contents
%\begin{abstract}
Im Rahmen dieser Arbeit erfolgt eine Untersuchung des Themas \textbf{"`Klassifizierung von Hirntumoren mit Algorithmen des Deep Learning"'}. Das Ziel dieser Untersuchung ist die Analyse der Performance von \textbf{\acf{cnn}} bei der Klassifizierung von \acf{ht} anhand von \acf{mrt}-Bildern.\\
Die zentrale Fragestellung lautet: Welcher Unterschied besteht in der Performance zwischen einem selbst entwickelten \ac{cnn}-Modell und einem vortrainierten Modell bei der Klassifizierung von \acs{ht}. Dabei wird evaluiert, ob ein durch \textbf{\acf{ft}} an die \ac{mrt}-Daten optimiertes vortrainiertes \ac{cnn}-Modell eine höhere Klassifikationsgenauigkeit erzielt als ein neu entwickeltes und vollständig trainiertes \ac{cnn}. Im Rahmen des \ac{ft} erfolgt die Feinabstimmung eines vortrainierten Modells auf einen spezifischen Datensatz durch Anpassung tiefer Schichten.\\
Die Hypothese besagt, dass \textbf{\acf{tl}}, bei dem ein Modell, das zuvor auf einem umfangreichen Datensatz wie \textbf{ImageNet} trainiert wurde, in Kombination mit \ac{ft} eine signifikante Verbesserung der Klassifikationsleistung bewirkt.\\
Zur Beantwortung der Fragestellung wird der öffentlich zugängliche \ac{ds} \citet[]{Nickparvar:2021} verwendet, der \textbf{7.023 \ac{mrt}-Bilder} verschiedener \ac{ht} sowie nicht tumoröser Gewebe enthält. Die Methodik beinhaltet sowohl die Implementierung eines eigenen \ac{cnn} als auch die Anpassung des vortrainierten ResNet-50-Modells von \citet[]{He:2016} an dem Datensatz. Die Modelle werden anhand von Metriken wie \textbf{Accuracy}, \textbf{Recall}, \textbf{Preciscion} und \textbf{F1-Score} evaluiert, ergänzt durch \textbf{Konfusionsmatrizen}. Die Implementierung wird in \textbf{Python} unter Verwendung von Bibliotheken wie \textbf{TensorFlow} und \textbf{Keras} durchgeführt.\\
Die Arbeit ist in mehrere Abschnitte gegliedert. Nach der Einführung in das Thema sowie einem Literaturüberblick erfolgt die Erläuterung des theoretischen Hintergrunds der Modelle und der Evaluierungsmethoden. Der methodische Teil beschreibt den Datensatz und die Modellierungsansätze begleitet von Architekturdiagrammen. Daraufhin werden die Resultate präsentiert und analysiert sowie im Kontext der formulierten Forschungsfragen und Hypothesen diskutiert. Abschließend werden die wesentlichen Erkenntnisse zusammengefasst und künftige Forschungsperspektiven in diesem Bereich erörtert.\\
Das Literaturverzeichnis enthält aktuelle, relevante und zentrale wissenschaftliche Artikel und Bücher, die als theoretische und methodische Grundlage dienen.
%\end{abstract} % Zur besseren Strukturierung empfiehlt es sich das Dokument aufzuteilen.
\newpage

%%%%%%%%%%%%%%%%%%%%%%%%%%%%%%%%%%%%%%%%%%%%%%%%%%%%%%%%%%%%%%%%%%%%%%%
%%%%%%%%%%%%%%%%%%%%%%%%%%%%%%%%%%%%%%%%%%%%%%%%%%%%%%%%%%%%%%%%%%%%%%%
%%%%%%%%%%%%%%%%%%%%%%%%%%%%%%%%%%%%%%%%%%%%%%%%%%%%%%%%%%%%%%%%%%%%%%%

\pagenumbering{arabic}
\onehalfspacing 			     % Für richtigen Zeilenabstand

%\section*{Disposition}
\addcontentsline{toc}{section}{Disposition}
% add the unnumbered section to the table of contents
%\begin{abstract}
Im Rahmen dieser Arbeit erfolgt eine Untersuchung des Themas \textbf{"`Klassifizierung von Hirntumoren mit Algorithmen des Deep Learning"'}. Das Ziel dieser Untersuchung ist die Analyse der Performance von \textbf{\acf{cnn}} bei der Klassifizierung von \acf{ht} anhand von \acf{mrt}-Bildern.\\
Die zentrale Fragestellung lautet: Welcher Unterschied besteht in der Performance zwischen einem selbst entwickelten \ac{cnn}-Modell und einem vortrainierten Modell bei der Klassifizierung von \acs{ht}. Dabei wird evaluiert, ob ein durch \textbf{\acf{ft}} an die \ac{mrt}-Daten optimiertes vortrainiertes \ac{cnn}-Modell eine höhere Klassifikationsgenauigkeit erzielt als ein neu entwickeltes und vollständig trainiertes \ac{cnn}. Im Rahmen des \ac{ft} erfolgt die Feinabstimmung eines vortrainierten Modells auf einen spezifischen Datensatz durch Anpassung tiefer Schichten.\\
Die Hypothese besagt, dass \textbf{\acf{tl}}, bei dem ein Modell, das zuvor auf einem umfangreichen Datensatz wie \textbf{ImageNet} trainiert wurde, in Kombination mit \ac{ft} eine signifikante Verbesserung der Klassifikationsleistung bewirkt.\\
Zur Beantwortung der Fragestellung wird der öffentlich zugängliche \ac{ds} \citet[]{Nickparvar:2021} verwendet, der \textbf{7.023 \ac{mrt}-Bilder} verschiedener \ac{ht} sowie nicht tumoröser Gewebe enthält. Die Methodik beinhaltet sowohl die Implementierung eines eigenen \ac{cnn} als auch die Anpassung des vortrainierten ResNet-50-Modells von \citet[]{He:2016} an dem Datensatz. Die Modelle werden anhand von Metriken wie \textbf{Accuracy}, \textbf{Recall}, \textbf{Preciscion} und \textbf{F1-Score} evaluiert, ergänzt durch \textbf{Konfusionsmatrizen}. Die Implementierung wird in \textbf{Python} unter Verwendung von Bibliotheken wie \textbf{TensorFlow} und \textbf{Keras} durchgeführt.\\
Die Arbeit ist in mehrere Abschnitte gegliedert. Nach der Einführung in das Thema sowie einem Literaturüberblick erfolgt die Erläuterung des theoretischen Hintergrunds der Modelle und der Evaluierungsmethoden. Der methodische Teil beschreibt den Datensatz und die Modellierungsansätze begleitet von Architekturdiagrammen. Daraufhin werden die Resultate präsentiert und analysiert sowie im Kontext der formulierten Forschungsfragen und Hypothesen diskutiert. Abschließend werden die wesentlichen Erkenntnisse zusammengefasst und künftige Forschungsperspektiven in diesem Bereich erörtert.\\
Das Literaturverzeichnis enthält aktuelle, relevante und zentrale wissenschaftliche Artikel und Bücher, die als theoretische und methodische Grundlage dienen.
%\end{abstract} % Zur besseren Strukturierung empfiehlt es sich das Dokument aufzuteilen.
%\newpage

\section{Einführung}

In den vergangenen Jahren hat die rapide Entwicklung des \acf{ml} auf globaler Ebene zu einem Paradigmenwechsel geführt. Als besonders bemerkenswert erweist sich die Anwendung von \ac{ml} in der medizinischen Bildgebung, welche eine präzise und effiziente Diagnostik ermöglicht. Algorithmen des \acf{dl}, insbesondere \acf{cnn}, haben sich dabei als zentrale Technologie etabliert, da sie in der Lage sind, komplexe Muster in Bilddaten zu erkennen und zu klassifizieren. Diese Fortschritte eröffnen neue Perspektiven für die Diagnostik von Krankheiten, die eine schnelle und zuverlässige Analyse erfordern. (Vgl. \citealt[Vorwort]{Deprez:2023}). 
\\
\acf{ht} stellen ein erhebliches medizinisches Herausforderungsgebiet dar. Weltweit wird ihre Inzidenzrate mit 3,5 und die Mortalitätsrate mit 2,6 pro 100.000 Menschen angegeben (vgl. \citealt[]{Ferlay:2024}). Eine präzise Differenzierung zwischen den häufigsten Arten wie Gliomen (engl. \textit{glioma}), Meningeomen (engl. \textit{meningioma}) und Hypophysentumoren (engl. \textit{pituitary}), ist von grundlegender Bedeutung für die Auswahl einer geeigneten Therapie (vgl. \citealt[]{Bilsky:2023_}). Die Abbildung \ref{fig:1} veranschaulicht eine Auswahl von \ac{mrt}-Aufnahmen der drei \ac{ht}-Arten sowie eine \ac{mrt}-Aufnahme nicht tumorösen Gewebes (engl. \textit{no tumor}):
\begin{figure}[ht]
    \centering
        \includegraphics[width=0.65\textwidth]{BA/Abbildungen/Aufnahmearten.jpg}
        \caption{Auswahl von \ac{mrt}-Bildern der vier Aufnahmearten.}
    \label{fig:1}
\end{figure}\\
Die Diagnose von Hirntumoren basiert traditionell auf manuellen Analysen von Aufnahmen der \acf{mrt} und \acf{ct} durch Radiologen (vgl. \citealt[]{Oezkaraca:2023}). Der manuellen Überprüfung medizinischer Bilder ist jedoch ein hoher Zeitaufwand verbunden, zudem ist sie aufgrund des Patientenflusses potenziell fehleranfällig. Um dieses Problem zu lösen, ist die Entwicklung und Anwendung eines automatischen Bildanalyseverfahrens erforderlich, welches die Arbeitsbelastung bei der Klassifizierung und Diagnose von \ac{mrt}-Bildern des Gehirns reduziert, als Hilfsmittel für Radiologen und Ärzte dient und die Mortalität durch die Früherkennung von Tumoren senken kann (vgl. \citealt[Vorwort]{Deprez:2023}).
\\
In diesem Kontext hat sich \ac{dl} als vielversprechende Technologie erwiesen. \ac{cnn}s, die bereits bei Bildklassifikationsaufgaben wie der \ac{ilsvrc} \citet{Krizhevsky:2012} signifikante Erfolge erzielt haben, bieten auch im medizinischen Bereich ein beträchtliches Potenzial (vgl. \citealt[S. 103]{Lundervold:2019}). Die Fähigkeit zur automatischen Merkmalsextraktion stellt einen zentralen Vorteil von \ac{cnn} dar. Im Gegensatz zu klassischen Methoden erfordert ihre Architektur keine manuelle Feature-Definition, sondern extrahiert komplexe Merkmale direkt aus den Daten (vgl. \citealt[S. 105]{Lundervold:2019}). Dadurch sind sie für Aufgaben der medizinischen Bildverarbeitung geeignet, bei denen relevante Muster oft subtil und mehrdimensional sind.
\\
Der Einsatz von \ac{cnn}s ist jedoch mit Herausforderungen verbunden. Sie weisen einen hohen Rechenaufwand auf und sind insbesondere bei kleineren Datensätzen, die in der medizinischen Domäne häufig auftreten, anfällig für eine zu starke Anpassung an die Trainingsdaten. Dies kann zu einer eingeschränkten Generalisierungsfähigkeit auf neue Daten führen. (Vgl. \citealt[S. 28; 456]{Geron:2019}). 
\\
Vortrainierte \ac{cnn}-Modelle, die im Rahmen des \acf{tl} verwendet werden, stellen eine flexible Alternative dar. Sie ermöglichen die Übertragung bereits erlernter Merkmale aus großen Bilddatensätzen wie ImageNet \citet{Deng:2009} auf spezifische medizinische Aufgaben, wodurch der Bedarf an großen Datensätzen und Trainingsressourcen reduziert wird. Obwohl dies den Trainingsaufwand reduziert, könnte die Anpassungsfähigkeit solcher Modelle auf domänenspezifische Daten eingeschränkt sein. (Vgl. \citealt[S. 345-347]{Geron:2019}).
\\
Die vorliegende Arbeit befasst sich mit der Untersuchung des Einsatzes von \ac{cnn}s zur Klassifikation von \ac{ht} auf \ac{mrt}-Bildern. In diesem Kontext erfolgt eine Evaluierung eines selbst entwickelten, optimierten Modells sowie des vortrainierten Modells ResNet-50 von \citet{He:2016}. Ziel ist der Vergleich und die Analyse ihrer Performances anhand von verschiedenen Metriken. 
\\ 
\clearpage

\section{Literaturüberblick}

Der Forschungsstand zur Erkennung von \ac{ht} mittels \ac{cnn}s zeigt eine Vielzahl an Ansätzen, die gezielt auf spezifische Herausforderungen und Datensätze abgestimmt sind.  \citet{Seetha:2018} demonstrierten die Effektivität vortrainierter Modelle mit eingefrorenen tiefen Schichten. Diese erwiesen sich im Vergleich zu traditionellen Modellen wie \ac{dnn} und \ac{svm} als überlegen.
\\
In einer weiteren Studie untersuchten \citet{Kabir:2018} die Klassifikation verschiedener Gliomgrade. Dazu kombinierten sie \ac{cnn}s mit \ac{ga}, um die Leistung der Modellarchitektur zu optimieren. \citet{Kebir:2018} evaluierten \ac{cnn}s in Kombination mit einem k-Means-Algorithmus zur Klassifikation und Segmentierung eines klinischen \ac{ds}.
\\
\citet{Anaraki:2019} integrierten \ac{ga} in den Trainingsprozess, um die \ac{cnn}-Strukturen zu optimieren und konnten präzise Klassifikationsergebnisse erzielen. \citet{Deepak:2019} hingegen verfolgten einen Ansatz des \ac{tl}, wobei GoogleNet als Feature-Extraktor für die Klassifikation mit kleinerem \ac{ds} verwendet wurde.
\\
\citet{Ismael:2020} konnten die Genauigkeit durch den Einsatz von Residual Networks (ResNet50) und einer Datenaugmentation verbessern. \citet{Naser:2020} demonstrierten das Potential von U-Net und VGG16 für die Segmentierung und Klassifikation niedriggradiger Gliome (LGG).
\\
\citet{Mehrotra:2020} untersuchten verschiedene vortrainierte Modelle und optimierten die Klassifikationsleistung durch \ac{tl}. \citet{Sharif:2020} kombinierten \ac{dl} mit Feature-Fusion-Techniken und optimierten die Ergebnisse mit \ac{pso}. Ein ensemble-basierter Ansatz wurde von \citet{Tandel:2021} entwickelt, welcher sowohl \ac{dl}- als auch \ac{ml}-Modelle integrierte, um eine Mehrheitswahl-basierte Klassifikation zu erreichen.
\\
In ihrer Untersuchung setzten \citet{Gab:2021} VGG19 als Feature-Extraktor ein und nutzten Bildaugmentationstechniken wie PGGAN. \citet{Dutta:2022} führten mit CDANet eine duale Aufmerksamkeitsstruktur ein, um fokussierte Merkmale aus \ac{mrt}-Bildern mit höherer Präzision zu erfassen.
\\
\citet{Altahhan:2023} entwickelten hybride \ac{cnn}-Modelle, welche \ac{dl}- und \ac{ml}-Klassifika-toren kombinierten. Die Optimierung von AlexNet in Verbindung mit SVM und KNN zielte auf eine Erhöhung der Genauigkeit ab. In einer späteren Studie präsentierten \citet{Ravinder:2023} ein graphbasiertes \ac{cnn}-Modell (GCNN), welches die Berücksichtigung nicht-euklidischer Distanzen in Bilddaten ermöglicht.
\\
Die in Tabelle \ref{tab:2.1} sowie in Tabelle \ref{tab:2.2} zusammengefassten Studien verdeutlichen die Anpassungsfähigkeit von \ac{cnn}-Modellen an spezifische Anforderungen, einschließlich \ac{ga}, hybrider Architekturen und \ac{tl}. Sie veranschaulichen sowohl Fortschritte als auch bestehende Herausforderungen in der medizinischen Bildverarbeitung.
\begin{landscape}
\renewcommand{\arraystretch}{2.5}
\begin{table}[ht]
\centering
%\resizebox{1.47\textwidth}{!}{%
\begin{tabular}{||c||c|c|c|c|c||}
\hline
\textbf{Autor(Jahr)}  & \textbf{Aufgabe} & \textbf{Daten} & \textbf{Modellarchitektur}
& \textbf{Accuracy} & \textbf{\makecell{Anmerkungen}}
\\ \hline \hline
% 1  https://dx.doi.org/10.13005/bpj/1511
\citet{Seetha:2018} & Klassifikation   & \makecell{\citet{BRATS:2015} \\ \& \citeauthor{Radiopaedia:2024}}   &   \makecell{Vortrainierte \\ \ac{cnn}s}  &  97.5\%      & /     
\\ \hline \hline
% 2 [50]
\citet{Kabir:2018}  & \makecell{Klassifikation \& \\  Einstufung} & \makecell{REMBRANDT \\ \citet{Scarpace:2015}}& \makecell{Optimiertes \\\ac{cnn}} & 94.2\% &  \ac{ga} 
\\ \hline \hline
% 3 [5]
\citet{Kebir:2018}  & \makecell{Klassifikation \& \\ Segmentierung} & privater klinischer \ac{ds} & \makecell{Optimiertes \\\ac{cnn}} & 95\% & K-Mean Algorithmus 
\\ \hline \hline
% 4 https://doi.org/10.1016/j.bbe.2018.10.004
\citet{Anaraki:2019}  & \makecell{Klassifikation \& \\ Einstufung}  &    \makecell{\citet{IXI:2015}, \\ TCIA \citet{Clark:2013}, \\ \makecell{REMBRANDT \\ \citet{Scarpace:2015},}\\ \makecell{TCGA-GBM \\ \citet{Scarpace:2016}} \\ \makecell{\&  TCGA-LGG \\ \citet{Pedano:2016}}}     &    \makecell{Optimiertes \\\ac{cnn}} &  \makecell{90.9\% \\ 94.2\%}   & \ac{ga}                       
\\ \hline \hline
% 5 survey
\citet{Deepak:2019}    &   \makecell{Klassifikation }   &   Figshare \citet{Cheng:2017}  &   GoogleNet      &  93\%   &    /                    
\\ \hline \hline
% 6 survey
\citet{Ismael:2020}  & \makecell{Klassifikation }  &   Figshare \citet{Cheng:2017}  &   ResNet-50  &  97\%   &      Datenaugmentation            
\\ \hline \hline
% 7 survey
\citet{Naser:2020}     &  \makecell{Segmentierung  \& \\ Einstufung} &  TCIA \citet{Clark:2013}  & \makecell{U-net \\ \& VGG-16 }  &  \makecell{92.0\% \\ 89.0\%} &        /     \\ \hline \hline
\end{tabular}%
%}
\caption{Überblick über relevante Arbeiten zur Erkennung von Hirntumoren mittels \ac{cnn}s I.}
\label{tab:2.1}
\end{table}
\end{landscape}





\newpage
\begin{landscape}
\renewcommand{\arraystretch}{3.4}
\begin{table}[ht]
\centering
\resizebox{0.98\linewidth}{!}{%
\begin{tabular}{||c||c|c|c|c|c||}
\hline
\textbf{Autor(Jahr)}  & \textbf{Aufgabe} & \textbf{Daten} & \textbf{Modellarchitektur}
& \textbf{Accuracy} & \textbf{\makecell{Anmerkungen}}
\\ \hline \hline
% 8 survey
\citet{Mehrotra:2020}   &   Klassifikation   & TCIA \citet{Clark:2013} &  \makecell{AlexNet, GoogLeNet \\ ResNet-50, ResNet-101 \\  \& SqueezeNet}  &   \makecell{96.8\%  \\ (AlexNet)
}  &      Binäre Klassifikation   
\\ \hline \hline
% 9 survey
\citet{Sharif:2020}    &   \makecell{Klassifikation \& \\  Segmentierung}    &  \makecell{\citet{BRATS:2013}, \\ \citet{BRATS:2014} \\ BRATS \citet{Bakas:2017} \\ \& BRATS \citet{Bakas:2018}} & Inception V3 &   \makecell{98.3\%,\\ 97.8\%, \\ 96.9\% \\ \& 92.5\%}   &   \ac{pso}                
\\ \hline \hline
% 10 survey
\citet{Tandel:2021}     &   Einstufung   & \makecell{REMBRANDT \\ \citet{Scarpace:2015}} &  \makecell{AlexNet, VGG-16, \\ ResNet-18, ResNet-50 \\ \& GoogleNet}   &   97.1\%   &  \makecell{Mehrheitswahl-basierter \\Ensemble-Algorithmus }                       
\\ \hline \hline
% 11 10.3390/diagnostics11122343
\citet{Gab:2021}     &    Klassifikation  & \makecell{privater klinischer \ac{ds} \\ \citet{Cheng:2015}} & VGG-19 &   98.5\%   &   \makecell{PGGAN \\ Datenaugmentation}    
\\ \hline \hline
% 12 survey 
\citet{Dutta:2022}   &   Klassifikation   &  Figshare \citet{Cheng:2017} &  CDANet &  96.7\%    &      /                             
\\ \hline \hline
% 13  10.3390/diagnostics13050864
\citet{Altahhan:2023}   &    Klassifikation  &  Kaggle \citet{Nickparvar:2021} & \makecell{AlexNet-KNN \\ \& AlexNet-\acs{svm}} &   \makecell{95.0\% \\ 97.0\%}   &  Hybride \ac{dl}-\ac{ml}-Modelle             
\\ \hline \hline
% 14 10.1038/s41598-023-41407-8
\citet{Ravinder:2023}    &   Klassifikation    & Kaggle \citet{Bhuvaji:2020} & GCNN &   95.0\%   &       Graphbasiertes Modell           
\\ \hline
\end{tabular}%
}
\caption{Überblick über relevante Arbeiten zur Erkennung von Hirntumoren mittels \ac{cnn}s II.}
\label{tab:2.2}
\end{table}
\end{landscape}
\newpage

\section{Theorie hinter Deep Learning}

Das \ac{dl} stellt einen spezialisierten Teilbereich des \ac{ml} dar, welcher auf der Verwendung tiefer neuronaler Netzwerke basiert. Die theoretischen Grundlagen für \ac{dl} , insbesondere im Bereich neuronaler Architekturen und Lernalgorithmen, wurden von \citet[S. 22-40]{Lang:2023} umfassend behandelt. In den folgenden Unterkapiteln (3.1 bis 3.9) wird ein Großteil dieser Ausführungen aufgegriffen und systematisch erläutert. Die zugrundeliegenden Formeln stammen hauptsächlich aus \citeauthor{Lang:2023}s Werk. Eine Ausnahme bildet Unterkapitel 3.5, in dem die Formeln und Ansätze auf den Arbeiten von \citet[]{Tiwari:2022} sowie \citet[]{Grandini:2020} basieren.

\subsection{Tiefe neuronale Netze}

Der grundlegende Ansatz hinter \ac{dl}-Algorithmen basiert auf der Idee, künstliche neuronale Netzwerke zu entwickeln, die die Funktionsweise des menschlichen Gehirns nachahmen (vgl. \citealt[S. 279]{Geron:2019}). Dieses Konzept wurde von \citet[]{Rosenblatt:1958} durch das Perzeptron-Modell eingeführt, welches als probabilistisches Modell für die Informationsspeicherung und -organisation im Gehirn dient. Die Bildung von Verbindungen zwischen künstlichen Neuronen erfolgt in ähnlicher Weise wie die neuronale Kommunikation im Gehirn, nämlich durch wiederholtes Training und entsprechende Anpassung der Gewichtungen (vgl. \citealt[S. 286-287]{Geron:2019}).\\
Ein künstliches Neuron multipliziert einen Eingangsvektor \(
\mathbf{x} = (x_1, x_2, \ldots, x_n)
\) mit einem Gewichtungsvektor \(
\mathbf{w} = (w_1, w_2, \ldots, w_n)
\), um ein skalares Signal zu bilden und addiert ein Bias-Term \(b\), um ein Logit zu erzeugen, das durch eine Aktivierungsfunktion \(f\) in das Ausgabesignal 
\begin{equation}
\begin{aligned} 
y(\mathbf{x}; \mathbf{w}, b) = f(\mathbf{x} \cdot \mathbf{w} + b)
\end{aligned} 
\end{equation}
umgewandelt wird.\\
Rosenblatts Modell eines künstlichen Neurons, das Perzeptron, verwendete die Heaviside-Stufenfunktion als Aktivierungsfunktion und einen negativen Bias-Term, um eine binäre Ausgabe zu erzeugen:
\begin{equation}
\begin{aligned}
y = \begin{cases}
1 & \text{wenn} \sum_{j} x_j \cdot w_j > b \\
0 & \text{wenn} \sum_{j} x_j \cdot w_j \leq b.
\end{cases}
\end{aligned} 
\end{equation}
Die Lösung komplexer Modellierungsprobleme erfordert die Verwendung komplexer Modelle, die über die Fähigkeiten eines einzelnen Perzeptrons hinausgehen. Eine Möglichkeit der Modellierung ist die Kombination mehrerer Perzeptrons zu einem Netzwerk, welches als \ac{mlp} bezeichnet wird. Layer, welche nicht direkt mit der Eingabe oder Ausgabe verbunden sind, werden als versteckte Schichten (engl. \textit{hidden layer}) bezeichnet. Modelle, die zwei oder mehr versteckte Schichten aufweisen, werden als tiefe neuronale Netze (engl. \textit{\ac{dnn}}) bezeichnet. Die Neuronen der ersten versteckten Schicht interagieren direkt mit den Eingabemerkmalen, während spätere Schichten fortgeschrittene Mustererkennung ermöglichen. Dies führt zu einer progressiv komplexeren und präziseren Entscheidungsfindung. (Vgl. \citealt[S. 25]{Lang:2023}).\\
In einem \ac{mlp} ist jedes Neuron einer Schicht mit allen Neuronen der vorhergehenden und der nachfolgenden Schicht verbunden. Diese Art von Schicht wird als dicht (engl. \textit{dense layer}) oder vollständig verbunden (engl. \textit{fully connected}) bezeichnet. (Vgl. \citealt[S. 285]{Geron:2019}). \\ 
Um das Netz zu trainieren und die optimalen Gewichtungen zu finden, wird ein Algorithmus verwendet, der die Gewichtungen schrittweise anpasst. Hierzu werden differenzierbare, kontinuierliche Aktivierungsfunktionen benötigt, da die Heaviside-Funktion nicht differenzierbar ist und somit ungeeignet wäre. Ein Netzwerk, das ausschließlich lineare Aktivierungsfunktionen verwendet, würde auf ein lineares Modell reduziert, wodurch seine Fähigkeit, komplexe Daten zu modellieren, stark eingeschränkt wäre. Daher weisen versteckte Schichten nichtlineare Aktivierungen auf, die es dem Netzwerk ermöglichen, komplexe Muster zu erkennen. (Vgl. \citealt[S. 26]{Lang:2023} sowie \citealt[S. 292]{Geron:2019}). \\
In Abbildung \ref{fig:4.1} werden gebräuchliche Aktivierungsfunktionen und ihre Derivate dargestellt, um ihre Auswirkungen auf die Ausgabewerte zu veranschaulichen. Diese Funktionen dienen primär dazu, Nichtlinearität in neuronale Netze einzuführen und somit komplexere Muster zu modellieren.\\
Eine typische Aktivierungsfunktion für versteckte Neuronen ist die ReLU (Rectified Linear Unit): 
\begin{equation}
\begin{aligned}
\phi(x) = \max(0, x).
\end{aligned}
\end{equation}
Die ReLU-Funktion erweist sich aufgrund ihrer Kombination von Einfachheit und Effizienz als besonders geeignet für die Verarbeitung großer Netzwerke. Die Funktion setzt negative Werte auf null und lässt positive Werte unverändert, wodurch eine schnelle Berechnung sowie ein sparsames Speicherverhalten ermöglicht werden. Bei tiefen Netzwerken führt die Deaktivierung negativer Eingaben zu einer Steigerung der Leistungsfähigkeit. (Vgl. \citealt[S. 292]{Geron:2019}).\\
Die Wahl der Aktivierungsfunktion in der Ausgabeschicht hängt vom Modellierungsproblem ab. Für die binäre Klassifikation wird typischerweise die Sigmoidfunktion verwendet: 
\begin{equation}
\begin{aligned}
    \sigma(z) = \frac{1}{1 + \exp(-z)}.
\end{aligned}
\end{equation}
Diese Funktion transformiert die Ausgabewerte in den Bereich zwischen 0 und 1, was sie ideal für Wahrscheinlichkeitsvorhersagen bei binären Problemen macht (vgl. \citealt[S. 143]{Geron:2019}).\\
Während bei Klassifizierungsproblemen mit \textit{K} verschiedenen Klassen häufig die Softmax-Funktion verwendet wird: 
\begin{equation}
\begin{aligned}
    \sigma(\mathbf{z})_i = \frac{\exp(z_i)}{\sum_{j=1}^{K} \exp(z_j)}.
\end{aligned}
\end{equation}
Die Softmax-Funktion normalisiert die Ausgaben, sodass sie eine Wahrscheinlichkeitsverteilung über die \textit{K} Klassen darstellen, wobei die Summe der Wahrscheinlichkeiten den Wert 1 annimmt. Diese Eigenschaft qualifiziert sie insbesondere für die Lösung von Mehrklassenklassifikationsproblemen, da jede Ausgabe als Wahrscheinlichkeit für eine Klasse interpretiert werden kann. (Vgl. \citealt[S. 148-149]{Geron:2019}).

\begin{figure}[H]
    \centering
    \begin{tikzpicture}
        \begin{axis}[
            width=0.45\textwidth,
            height=0.45\textwidth,
            axis lines=middle,
            xlabel={},
            ylabel={},
            ymin=-1.2, ymax=1.2,
            xmin=-5, xmax=5,
            title= {Aktivierungsfunktionen}
        ]
        % Step function
        \addplot[color=orange, thick] coordinates {
            (-5, -1) (0, -1) (0, 1) (5, 1)};
        % Sigmoid function
        \addplot[color=green, thick] {1 / (1 + exp(-x))};
        % Tanh function
        \addplot[dashdotted, color=cyan, thick] {tanh(x)};
        % ReLU function
        \addplot[thick, color=magenta, dashed] {max(0, x)};
        \end{axis}
    \end{tikzpicture}
    \hspace{0.5cm}
    \begin{tikzpicture}
        \begin{axis}[
            width=0.45\textwidth,
            height=0.45\textwidth,
            axis lines=middle,
            xlabel={},
            ylabel={},
            ymin=-0.2, ymax=1.2,
            xmin=-5, xmax=5,
            legend pos= north west,
            title={Derivate},
           legend style={at={(1.05,0.5)}, anchor=west}
        ]
        % Derivative of Step function
        \addplot[color=orange, thick] coordinates {
            (-5, 0) (0, 0) (0, 0) (0.0, 0) (5, 0)
        };
        % Derivative of Sigmoid function
        \addplot[ color=green, thick] {exp(-x) / ((1 + exp(-x))^2)};
        % Derivative of Tanh function
        \addplot[dashdotted, color=cyan, thick] {1 - tanh(x)^2};
        % Derivative of ReLU function
        \addplot[thick, dashed, color=magenta] coordinates {
            (-5, 0) (0, 0) (0, 1) (5, 1)
        };
        \addplot[red, only marks, mark=x] coordinates {(0, 0)}; 
        \legend{Step, Sigmoid, Tanh, ReLU}
        \end{axis}
    \end{tikzpicture}
    \caption{Aktivierungsfunktionen und deren Derivate, nach \citet[S. 292]{Geron:2019}.}
    \label{fig:4.1}
\end{figure}


\subsection{Backpropagation \& Gradientenabstieg}

Die Gewichte eines \ac{dnn} werden durch die Minimierung einer Verlustfunktion (engl. \textit{loss function}) angepasst, um möglichst genaue Vorhersagen zu erzielen. Die Verlustfunktion ist entscheidend für die Optimierung der Modellparameter. (Vgl. \citealt[S. 113]{Geron:2019}). Eine der grundlegenden Verlustfunktionen ist der mittlere quadratische Fehler (engl. \textit{\ac{mse}}): 
\begin{equation}
\begin{aligned}
\label{eq:4.6}
L_{\text{MSE}} = \frac{1}{2n} \sum_{x} \left\| y(x) - \tilde{y} \right\|^2,
\end{aligned}
\end{equation}
wobei \textit{n} die Anzahl der Trainingsbeispiele \(x\) und \( \tilde{y} \) die zugehörigen tatsächlichen Labels darstellt, während \( y(x) \) die Netzwerkausgabe repräsentiert.\\
Der Trainingsprozess beginnt mit der zufälligen Initialisierung der Gewichte, gefolgt von einer iterative Anpassung durch einen Optimierungsprozess zur Minimierung der Verlustfunktion (vgl. \citealt[S. 118]{Geron:2019}). Laut \citet[S. 83-85]{Goodfellow:2017} ist der Gradientenabstieg (engl. \textit{\ac{gd}}), der erstmals von \citet[]{Cauchy:1847} eingeführt wurde, einer der grundlegendsten Optimierungsalgorithmen. Der Algorithmus passt die Netzwerkparameter \(\theta\) entlang des negativen Gradienten der Verlustfunktion an. Die Größe der Anpassung wird durch die Lernrate \(\epsilon\) bestimmt:
\begin{equation}
\begin{aligned}
    \Delta \theta \equiv - \epsilon \nabla_\theta L(\theta).
\end{aligned}
\end{equation}
Im Verlauf jedes Optimierungsschritts werden die Netzwerkparameter gemäß der folgenden Regel aktualisiert:
\begin{equation}
\begin{aligned}
    \theta \leftarrow \theta - \epsilon \nabla_\theta L(\theta).
\end{aligned}
\end{equation}
Bei Anwendung der \ac{mse}-Verlustfunktion \eqref{eq:4.6} ergibt sich die Anpassung der Netzwerkparameter wie folgt:
\begin{equation}
\begin{aligned}
\label{eq:4.9}
    \Delta \theta = - \epsilon \nabla_\theta \left( \frac{1}{2n} \sum_{x} \| y(x) - \tilde{y} \|^2 \right).
\end{aligned}
\end{equation}
In Gleichung \eqref{eq:4.9} wird die Summe über alle Trainingsbeispiele berechnet, weshalb jedes Eingangssignal durch das Netz geleitet werden muss, um die Ausgangssignale \( y(x) \) zu berechnen. Erst im Anschluss kann eine Anpassung der Gewichte durch Gradientenabstieg erfolgen. Insbesondere bei großen Datensätzen ist dies ein rechnerisch aufwändiger Prozess. (Vgl. \citealt[S. 28]{Lang:2023}).\\
Dieses Vorgehen wird durch den Backpropagation-Algorithmus unterstützt, welcher durch \citet[]{Rumelhart:1986} und \citet[]{Rumelhart:1986_} entwickelt wurde. Der Prozess lässt sich in zwei Phasen unterteilen: Im ersten Schritt, dem sogenannten \textit{"`Forward Pass"'}, wird jede Trainingsinstanz durch das Netzwerk geleitet, um Vorhersagen zu treffen. Im sogenannten \textit{"`Reverse Pass"'} wird der durch den Vergleich der Vorhersage mit den tatsächlichen Werten entstandene Fehler durch die Schichten des Netzwerks zurückpropagiert. Auf diese Weise wird der Beitrag jeder Verbindung zum Fehler bestimmt, um die entsprechenden Gewichte anzupassen. Dieser Prozess wird als Gradientenabstiegsschritt (engl. \textit{Gradient Descent step}) bezeichnet, bei dem die Gewichte in kleinen Schritten optimiert werden, um den Fehler zu minimieren. (Vgl. \citealt[S. 291]{Geron:2019}).\\
Der hier beschriebene Algorithmus, der alle Trainingsdaten für jeden Optimierungsschritt verwendet, wird als Batch-Gradientenabstieg (engl. \textit{Batch Gradient Descent}) bezeichnet (vgl. \citealt[S. 122]{Geron:2019}). Obwohl dieser Ansatz präzise ist, kann er bei großen Datensätzen ineffizient sein und führt oft dazu, dass das Modell in lokalen Minima hängen bleibt (vgl. \citealt[S. 119]{Geron:2019}). Daher wurde der Mini-Batch-Gradientenabstieg (engl. \textit{Mini-batch Gradient Descent}) entwickelt, bei dem lediglich ein Teil der Daten in jedem Schritt Berücksichtigung findet, was den Rechenaufwand reduziert und zu einer stabileren Konvergenz führen kann (vgl. \citealt[S. 124-127]{Geron:2019}). Die Anzahl der Beispiele, die den Algorithmen vor jedem Optimierungsschritt vorgelegt werden, die sogenannte Batchgröße (engl. \textit{batch size}), beeinflusst die Optimierungsleistung und senkt die Berechnungsanforderungen (vgl. \citealt[S. 326]{Geron:2019}). \\
Moderne Algorithmen wie der Adam-Optimierer fügen dem Gradientenabstieg einen Impulsterm hinzu, um vergangene Gradienteninformationen in die Parameteraktualisierung einfließen zu lassen. Der Aktualisierungsschritt erfolgt folgendermaßen:
\begin{equation}
\begin{aligned}
\mathbf{m} &\leftarrow \beta \mathbf{m} - \epsilon \nabla_\theta L(\theta) \\
\theta &\leftarrow \theta + \mathbf{m},
\end{aligned}
\end{equation}
wobei \(\mathbf{m}\) der Impulsterm ist und \(\beta\) ein Wert zwischen 0 und 1 darstellt, der die Gewichtung des Impulsterms bestimmt. Die Berücksichtigung früherer Gradienten ermöglicht eine effizientere und robustere Konvergenz als dies bei einfachen Gradientenabstiegsverfahren der Fall ist. (Vgl. \citealt[S. 351-352]{Geron:2019}).

\subsection{Modellparametrisierung \& Generalisierung}

Die Minimierung der Verlustfunktion auf den Trainingsdaten zielt darauf ab, das Modell zu optimieren, indem die Fehler auf den Trainingsdaten reduziert werden. Die Annahme basiert darauf, dass das Modell allgemeine Muster erlernt, welche auf das zugrunde liegende Problem übertragbar sind, sodass auch auf neuen, bisher nicht betrachteten Datensätzen adäquate Resultate erzielt werden. Diese Fähigkeit des Modells, auch auf neuen Daten eine gute Performance zu zeigen, wird als Generalisierung (engl. \textit{generalization}) bezeichnet. Ein niedriger Trainingsfehler (engl. \textit{training error}) indiziert eine effektive Performance des Modells auf den Trainingsdaten, während ein geringer Generalisierungsfehler (engl. \textit{generalization error}) darauf hinweist, dass das Modell seine Leistung auch auf unabhängigen Daten beibehalten kann. Allerdings besteht häufig eine gewisse Spannung zwischen den Zielen der Optimierung und der Generalisierung, da eine Minimierung des Trainingsfehlers nicht zwangsläufig zu einer Verbesserung der Generalisierung führt. Das Ziel ist daher, einen Ausgleich zu finden, um sowohl auf den Trainings- als auch auf den Testdaten gute Ergebnisse zu erzielen. (Vgl. \citealt[S. 128]{Chollet:2021}).\\
\ac{dnn} verfügen über eine signifikante Anzahl an trainierbaren Parametern, die in vielen Fällen eine Größe von Zehntausenden bis zu Millionen aufweisen. Die hohe Anzahl an Parametern ermöglicht dem Modell eine hohe Flexibilität bei der Erfassung und Verarbeitung komplexer Datensätze. (Vgl. \citealt[S. 364]{Geron:2019}). Diese Flexibilität ist auf die Freiheitsgrade (engl. \textit{\ac{dof}}) des Modells zurückzuführen, welche durch die Anzahl der Trainingsparameter definiert werden (vgl. \citet[S. 28]{Geron:2019}). Eine hohe Anzahl an Freiheitsgraden birgt jedoch das Risiko einer Überanpassung (engl. \textit{overfitting}), bei der das Modell spezifische Details der Trainingsdaten anstatt allgemeiner Muster erlernt. Dies kann zu einer Beeinträchtigung der Leistung des Modells auf neuen, ungesehenen Daten führen. (Vgl. \citealt[S. 110-111]{Chollet:2021}). Demgegenüber kann ein Modell, welches über eine unzureichende Anzahl an Freiheitsgraden verfügt, wesentliche Muster nicht adäquat erfassen, was als Unteranpassung (engl. \textit{underfitting}) bezeichnet wird. (Vgl. \citealt[S. 29]{Geron:2019}). Daher ist es von entscheidender Bedeutung, bei der Modellparametrisierung ein ausgewogenes Verhältnis zwischen der Optimierung der Leistung auf den Trainingsdaten und der Minimierung des Generalisierungsfehlers zu finden, um eine adäquate Leistung auf neuen Daten zu gewährleisten (vgl. \citealt[S. 28]{Geron:2019}).\\
Zur Überwachung des Generalisierungsfehlers des Modells während des Trainings wird ein Teil der Trainingsdaten als Validierungsset verwendet. Das unabhängige Set, welches nicht in die Modellanpassung einfließt, findet nach jedem Optimierungsschritt bzw. jeder Trainingsepoche Anwendung, um eine unvoreingenommene Verlustfunktion und weitere Metriken zu berechnen. (Vgl. \citealt[S. 102-104]{Chollet:2021}).\\
Die Generalisierungsdifferenz, definiert als die Abweichung der Modellleistung zwischen Trainings- und Validierungsset, erlaubt eine Beurteilung der Qualität der Generalisierung auf neue, bislang nicht betrachtete Daten. Eine hohe Abweichung kann auf eine Overfitting oder Underfitting hindeuten. Zur Optimierung der Modellleistung wird die Performance auf dem Validierungsset häufig zur Anpassung von Hyperparametern wie der Anzahl der Schichten, Neuronen oder der Lernrate genutzt, da diese direkt die Modellkapazität und Generalisierung beeinflussen. (Vgl. \citealt[S. 29]{Lang:2023}).\\
Für die abschließende Bewertung der Modellleistung wird schließlich ein drittes, unabhängiges Testset verwendet. Das Testset dient der finalen Evaluierung und wird nicht für Training oder Validierung verwendet. Dies gewährleistet eine objektive Beurteilung der Modellleistung. Daher basiert das Training auf drei Datensätzen: dem Trainingsset zur Anpassung der Gewichte, dem Validierungsset zur Überwachung der Generalisierung und dem Testset zur abschließenden Bewertung. (Vgl. \citealt[S. 151-152]{Chollet:2021}).

\subsection{Problemformulierung}

Die Auswahl einer passenden Verlustfunktion stellt einen wesentlichen Aspekt im Rahmen der Modellierung dar. Es existiert eine Vielzahl von Verlustfunktionen, die für spezifische Aufgabenbereiche adaptiert wurden. In der Praxis wird bei der Regression, bei der eine skalare Ausgabe modelliert wird, häufig die \ac{mse}-Verlustfunktion  (\ref{eq:4.6}) verwendet. Im Kontext des Trainings von Klassifikationsmodellen kommt hingegen in der Regel die kategoriale Kreuzentropie (engl. \textit{\ac{cce}}) zum Einsatz:
\begin{align}
    L_\text{CCE} = -\sum_{i=1}^{n} y_i \log(p_i),
\end{align}
wobei \(y_i\) für das wahre Label, \(p_i\) für die Softmax-Wahrscheinlichkeit der Klasse \(i\) und \(n\) für die Gesamtanzahl der Klassen steht. (Vgl. \citealt[S. 101-102]{Chollet:2021}). Die \ac{cce}-Verlustfunktion bestraft falsche Vorhersagen stärker als der \ac{mse},  weshalb sie sich insbesondere für Klassifikationsaufgaben eignet (vgl. \citealt[S. 149]{Geron:2019}).\\
Bei unbalancierten Datensätzen, deren Klassenverteilung eine unausgeglichene Verteilung aufweist, kann die Einführung eines Gewichtungstermins in die Verlustfunktion dazu beitragen, die Repräsentanz unterrepräsentierter Klassen zu verbessern. Dazu wird jeder Summand mit einem Gewichtungsfaktor \(w_i\) multipliziert, sodass unterrepräsentierte Klassen stärker berücksichtigt werden:
\begin{align}
    L_\text{wCCE} = -\sum_{i=1}^{n} w_i y_i \log(p_i).
\end{align}
(Vgl. \citealt[S. 30]{Lang:2023}).

\subsection{Leistungsbewertung} 
Neben der Verlustfunktion ist die Auswahl geeigneter Metriken für die Evaluierung der Modellleistung während der Validierung und des Tests von entscheidender Bedeutung. Diese Metriken spiegeln nicht nur die Genauigkeit des Modells wider, sondern auch dessen Fähigkeit, mit unausgewogenen Datensätzen umzugehen und auf unbekannte Daten zu generalisieren. (Vgl. \citet[S. 34-35]{Tiwari:2022}). \\
Eine Konfusionsmatrix (engl. \textit{\acf{cm}}) stellt ein wesentliches Instrument zur Darstellung der tatsächlichen Labels im Verhältnis zu den prognostizierten Labels dar. Die Darstellung erfolgt in Form einer Matrix, in der die Anzahl der korrekt vorhergesagten positiven Fälle (engl. \textit{\ac{tp}}), der korrekt vorhergesagten negativen Fälle (engl. \textit{\ac{tn}}), der falsch als positiv vorhergesagten Fälle (engl. \textit{\ac{fp}}) sowie der falsch als negativ vorhergesagten Fälle (engl. \textit{\ac{fn}}) angezeigt wird. Aus der \ac{cm} lassen sich mehrere Metriken ableiten, die zur Bewertung von Klassifikationsmodellen für binäre Klassen sowie für Mehrfachklassifikationen verwendet werden können. Die Abbildung \ref{fig:4.2} illustriert eine \ac{cm} für binäre Klassifikationen. Bei einer Mehrklassenklassifikation wie in der Abbildung \ref{fig:4.3} wird die Matrix um die Vorhersagen für jede Klasse gegenüber den tatsächlichen Klassen erweitert, wodurch eine detaillierte Analyse der Modellleistung ermöglicht wird. (Vgl. \citealt[S. 52]{Tiwari:2022}).
\begin{figure}[H]
    \centering
     % Resize the entire tikzpicture
    \resizebox{0.5\textwidth}{!}{ % Scale to 50% of text width
    \begin{tikzpicture}
        % Definiere die Farben für die Felder
        \definecolor{tpcolor}{RGB}{255, 255, 204}
        \definecolor{fncolor}{RGB}{255, 204, 204}
        \definecolor{fpcolor}{RGB}{204, 255, 204}
        \definecolor{tncolor}{RGB}{224, 224, 255}

        % Zeichne die Zellen
        \fill[tpcolor] (0, 0) rectangle (4, -4);
        \fill[fncolor] (4, 0) rectangle (8, -4);
        \fill[fpcolor] (0, -4) rectangle (4, -8);
        \fill[tncolor] (4, -4) rectangle (8, -8);

        % Zeichne die Begrenzungslinien
        \draw[thick] (0, 0) rectangle (8, -8);
        \draw[thick] (0, -4) -- (8, -4);
        \draw[thick] (4, 0) -- (4, -8);

        % Beschriftung der Zellen
        \node at (2, -2) {True Positive (TP)};
        \node at (6, -1.8) {False Negative (FN)};
        \node at (6, -2.3){\textcolor{purple}{Type II Error}}; 
        \node at (2, -5.8) {False Positive (FP)};
        \node at (2, -6.3){\textcolor{purple}{Type I Error}}; 
        \node at (6, -6) {True Negative (TN)};

        % Beschriftung der Achsen
        \node[rotate=90] at (-1, -4) {Tatsächliches Label};
        \node at (4, 1) {Prognostiziertes Label};
        
        % Beschriftung der Klassen
        \node at (2, 0.5) {Positive};
        \node at (6, 0.5) {Negative};
        \node[rotate=90] at (-0.5, -2) {Positive};
        \node[rotate=90] at (-0.5, -6) {Negative};
        
        % Überschrift der Matrix
       % \node at (4, 2) {\textbf{Confusion Metric}};

        % Pfeile für die Achsen
        %\draw[thick,->] (-1.5, -8) -- (-1.5, 0);
        %\draw[thick,->] (0, 1.5) -- (8, 1.5);
    \end{tikzpicture}
    }
    \caption{Binäre \ac{cm}, nach \citet[S. 52]{Tiwari:2022}.}
    \label{fig:4.2}
\end{figure}

\setlength{\leftmargini}{5pt} % Setzt die Einrückung auf 0
\begin{itemize}
  \item \textbf{\textit{Accuracy}}:\\
  Die Accuracy (Genauigkeit) eines Modells gibt das Verhältnis zwischen der Gesamtzahl der vom Modell richtig prognostizierten Labels und der Gesamtzahl der Labels an.
   \begin{equation}
  \begin{aligned}
      Accuracy = \frac{\text{TP} + \text{TN}}{\text{TP} + \text{TN} + \text{FP} + \text{FN}}
  \end{aligned}
  \end{equation}
  In Fällen einer unbalancierten Klassenverteilung kann die Accuracy jedoch irreführende Ergebnisse liefern, da sie durch überrepräsentierte Klassen dominiert werden kann (vgl. \citealt[S. 56]{Tiwari:2022}).
  \item \textbf{\textit{\ac{tpr}}}:\\
  Die \ac{tpr} beschreibt das Verhältnis der korrekt prognostizierten positiven Labels zur Gesamtzahl der tatsächlichen positiven Labels. Die \ac{tpr} ist auch als \textbf{\textit{Recall}} oder \textbf{\textit{Sensitivity}} bekannt.
   \begin{equation}
  \begin{aligned}
      TPR = \frac{\text{TP}}{\text{TP} + \text{FN}}
  \end{aligned}
  \end{equation}
   \item \textbf{\textit{\ac{tnr}}}:\\
  Die \ac{tnr} beschreibt das Verhältnis der korrekt prognostizierten negativen Labels zur Gesamtzahl der tatsächlichen negativen Labels. Eine andere Bezeichnung für \ac{tnr} ist die \textbf{\textit{Specificity}}. 
   \begin{equation}
  \begin{aligned}
      TNR = \frac{\text{TN}}{\text{FP} + \text{TN}}
  \end{aligned}
  \end{equation}
  Im Rahmen binärer Klassifikationsaufgaben, wie sie in zahlreichen klinischen Tests zum Einsatz gelangen, sind die Metriken Sensitivität und Spezifität von entscheidender Bedeutung. Sie stehen proportional zur Güte des Modells (vgl. \citealt[S. 30-31]{Lang:2023}). 
 \item \textbf{\textit{\ac{fnr}}}:\\
 Die \ac{fnr} stellt das Verhältnis der falsch vorhergesagten positiven Labels zur Gesamtzahl der tatsächlichen positiven Labels dar.
   \begin{equation}
   \begin{aligned}
      FNR = \frac{\text{FN}}{\text{TP} + \text{FN}}
  \end{aligned}
  \end{equation}
   \item \textbf{\textit{\ac{fpr}}}:\\
 Die \ac{fpr} gibt das Verhältnis der falsch vorhergesagten negativen Labels zur Gesamtzahl der tatsächlichen negativen Labels an.
    \begin{equation}
   \begin{aligned}
      FPR = \frac{\text{FP}}{\text{FP} + \text{TN}}
  \end{aligned}
  \end{equation}
  Die Güte des Modells steht in inversem Verhältnis zu den Metriken \ac{tnr} und \ac{fpr} (vgl. \citealt[S. 57-58]{Tiwari:2022}).
  \item \textbf{\textit{Precision}}:\\
 Die Precision gibt an, wie viele positive Labels im Verhältnis zu allen positiv prognostizierten Labels richtig prognostiziert wurden. 
    \begin{equation}
   \begin{aligned}
      Precision = \frac{\text{TP}}{\text{TP} + \text{FP}}
  \end{aligned}
  \end{equation}
  Es besteht eine negative Korrelation zwischen Precision und Recall (\ac{tpr}) (vgl. \citealt[S. 60-61]{Tiwari:2022}).
   \item \textbf{\textit{F1-Score}}:\\
Der F1-Score stellt ein harmonisches Mittel von Precision und Recall dar, welches eine Balance zwischen beiden Metriken bietet.\\
\begin{equation}
    \begin{aligned}
        F_1 &= \frac{2}{\text{Precision}^{-1} + \text{Recall}^{-1}}
        &= 2\cdot \frac{\text{Precision} \cdot \text{Recall}}{\text{Precision} + \text{Recall}}
    \end{aligned}
\end{equation}
 Der F1-Score gewichtet niedrige Werte deutlich stärker als hohe Werte. Folglich wird ein hoher F1-Score nur bei hohen Werten für Recall und Precision erreicht. (Vgl. \citealt[S. 61]{Tiwari:2022}).
\end{itemize}
Im Rahmen einer Mehrklassenklassifikation erfolgt die Berechnung von Precision, Recall und F1-Score für jede Klasse. Die Ergebnisse werden in einem Klassifikationsbericht (engl. \textit{\acf{cr}}) dargestellt. Die nachfolgende Tabelle \ref{tab:4.1} veranschaulicht dies anhand von einem \ac{cr} für \(n\) = 4 Klassen und insgesamt 1550 Datenpunkte. Der jeweilige Support gibt die Anzahl der beobachteten Datenpunkte bei der jeweiligen Metrik an.  Im Folgenden werden die verschiedenen Verfahren zur Berechnung der Metriken erläutert, nämlich Micro Average, Macro Average und Weighted Average. (Vgl. \citealt[S. 6]{Grandini:2020}).

% Definiere die Farben
\definecolor{tpcolor}{RGB}{255, 255, 204}
\definecolor{fncolor}{RGB}{255, 204, 204}
\definecolor{fpcolor}{RGB}{204, 255, 204}
\definecolor{tncolor}{RGB}{224, 224, 255}

% Rechte Tabelle: Konfusionsmatrix mit Werten
\begin{figure}[ht]
    \centering
     \resizebox{0.7\textwidth}{!}{ % Scale to 70% of text width
    \begin{tikzpicture}

        % Rechte Tabelle
        \node at (12, 4.5) {\textbf{Prognostiziertes Label}};
        \node[rotate=90] at (5, 1) {\textbf{Tatsächliches Label}};

        % Horizontale Kopfzeile
        \draw[thick, fill=white] (6, 3) rectangle (8, 4);
        \node at (7, 3.5) {\(i\)};
        \draw[thick, fill=white] (8, 3) rectangle (10, 4);
        \node at (9, 3.5) {1};
        \draw[thick, fill=Black!15] (10, 3) rectangle (12, 4);
        \node at (11, 3.5) {2};
        \draw[thick, fill=white] (12, 3) rectangle (14, 4);
        \node at (13, 3.5) {3};
        \draw[thick, fill=white] (14, 3) rectangle (16, 4);
        \node at (15, 3.5) {4};
        \draw[thick, fill=white] (16, 3) rectangle (18, 4);
        \node at (17, 3.5) {\(\sum\)};

        % Vertikale Kopfzeile
        \draw[thick, fill=white] (6, 2) rectangle (8, 3);
        \node at (7, 2.5) {1};
        \draw[thick, fill=Black!15] (6, 1) rectangle (8, 2);
        \node at (7, 1.5) {2};
        \draw[thick, fill=white] (6, 0) rectangle (8, 1);
        \node at (7, 0.5) {3};
        \draw[thick, fill=white] (6, -1) rectangle (8, 0);
        \node at (7, -0.5) {4};
        \draw[thick, fill=white] (6, -2) rectangle (8, -1);
        \node at (7, -1.5) {\(\sum\)};

        % Zellen - Zeile 1
        \draw[thick, fill=tncolor] (8, 2) rectangle (10, 3); % TN
        \node at (9, 2.5) {50};

        \draw[thick, fill=fpcolor] (10, 2) rectangle (12, 3); % FP
        \node at (11, 2.5) {37};

        \draw[thick, fill=tncolor] (12, 2) rectangle (14, 3); % TN
        \node at (13, 2.5) {24};

        \draw[thick, fill=tncolor] (14, 2) rectangle (16, 3); % TN
        \node at (15, 2.5) {39};

        \draw[thick, fill=white] (16, 2) rectangle (18, 3); % Zeilensumme
        \node at (17, 2.5) {150};

        % Zellen - Zeile 2
        \draw[thick, fill=fncolor] (8, 1) rectangle (10, 2); % FN
        \node at (9, 1.5) {10};

        \draw[thick, fill=tpcolor] (10, 1) rectangle (12, 2); % TP
        \node at (11, 1.5) {480};

        \draw[thick, fill=fncolor] (12, 1) rectangle (14, 2); % FN
        \node at (13, 1.5) {5};

        \draw[thick, fill=fncolor] (14, 1) rectangle (16, 2); % FN
        \node at (15, 1.5) {3};

        \draw[thick, fill=white] (16, 1) rectangle (18, 2); % Zeilensumme
        \node at (17, 1.5) {498};

        % Zellen - Zeile 3
        \draw[thick, fill=tncolor] (8, 0) rectangle (10, 1); % TN
        \node at (9, 0.5) {14};

        \draw[thick, fill=fpcolor] (10, 0) rectangle (12, 1); % FP
        \node at (11, 0.5) {10};

        \draw[thick, fill=tncolor] (12, 0) rectangle (14, 1); % TN
        \node at (13, 0.5) {765};

        \draw[thick, fill=tncolor] (14, 0) rectangle (16, 1); % TN
        \node at (15, 0.5) {1};

        \draw[thick, fill=white] (16, 0) rectangle (18, 1); % Zeilensumme
        \node at (17, 0.5) {790};

        % Zellen - Zeile 4
        \draw[thick, fill=tncolor] (8, -1) rectangle (10, 0); % TN
        \node at (9, -0.5) {0};

        \draw[thick, fill=fpcolor] (10, -1) rectangle (12, 0); % FP
        \node at (11, -0.5) {2};

        \draw[thick, fill=tncolor] (12, -1) rectangle (14, 0); % TN
        \node at (13, -0.5) {9};

        \draw[thick, fill=tncolor] (14, -1) rectangle (16, 0); % TN
        \node at (15, -0.5) {101};

        \draw[thick, fill=white] (16, -1) rectangle (18, 0); % Zeilensumme
        \node at (17, -0.5) {112};

        % Spaltensummen
        \draw[thick, fill=white] (8, -2) rectangle (10, -1);
        \node at (9, -1.5) {74};

        \draw[thick, fill=white] (10, -2) rectangle (12, -1);
        \node at (11, -1.5) {529};

        \draw[thick, fill=white] (12, -2) rectangle (14, -1);
        \node at (13, -1.5) {803};

        \draw[thick, fill=white] (14, -2) rectangle (16, -1);
        \node at (15, -1.5) {144};

        \draw[thick, fill=white] (16, -2) rectangle (18, -1);
        \node at (17, -1.5) {1550};

    \end{tikzpicture}
    }
 \resizebox{0.5\textwidth}{!}{ % Scale to 50% of text width
    % Legende
    \vspace{0.5cm}
    \begin{tikzpicture}
    
    \draw[fill=tpcolor] (1, -0.3) rectangle (1.8, 0.3);
    \node at (1.4, 0) {TP};
    
    \draw[fill=fncolor] (4, -0.3) rectangle (4.8, 0.3);
    \node at (4.4, 0) {FN};
    
    \draw[fill=fpcolor] (7, -0.3) rectangle (7.8, 0.3);
    \node at (7.4, 0) {FP};
    
    \draw[fill=tncolor] (10, -0.3) rectangle (10.8, 0.3);
    \node at (10.4, 0) {TN};
\end{tikzpicture}
}
    \caption{Mehrklassen-\ac{cm} mit Klasse \colorbox{Black!20}{2} als Bezugsklasse, nach \citet[S. 6]{Grandini:2020}.}
    \label{fig:4.3}
\end{figure}

% Classification Report
\begin{table}[H]
\centering
 \resizebox{0.7\textwidth}{!}{ % Scale to 70% of text width
\begin{tabular}{||c||c|c|c|c||}
\hline
\textbf{\(i\)} & \textbf{Precision} & \textbf{Recall} & \textbf{F1-Score} & \textbf{Support} \\
\hline \hline
1 & 0.68 & 0.33 & 0.44 & 150 \\\hline
2 & 0.91 & 0.96 & 0.94 & 498 \\\hline
3 & 0.95 & 0.97 & 0.96 & 790 \\\hline
4 & 0.70 & 0.90 & 0.79 & 112 \\\hline
\textbf{Accuracy} & \multicolumn{3}{c|}{0.90} &1550 \\\hline
\textbf{Macro Avg} & 0.81 & 0.79 & 0.78 & 1550 \\\hline
\textbf{Weighted Avg} & 0.89 & 0.90 & 0.89 & 1550 \\\hline
\end{tabular}
}
\caption{\ac{cr} basierend auf der \ac{cm} in Abbildung \ref{fig:4.3}.}
\label{tab:4.1}
\end{table}

\begin{itemize}
    \item \textbf{\textit{Micro Average}}:\\
    Der Micro Average berechnet die Gesamtperformance eines Modells über alle Klassen hinweg, indem die \ac{tp}, \ac{fp} und \ac{fn} aller Klassen summiert werden, bevor Precision, Recall und der F1-Score ermittelt werden. Diese Vorgehensweise gewährleistet eine gleichwertige Gewichtung aller Instanzen, unabhängig von der Klassengröße, was insbesondere bei ungleich großen Klassen von Vorteil ist. 
    \begin{equation}
    \begin{aligned}
        Micro\ Average\ Precision &= \frac{\sum_{i=1}^{n} \text{TP}_i}{\sum_{i=1}^{n} \text{Total Column}_i}
        &=\frac{\sum_{i=1}^{n} \text{TP}_i}{\text{Grand Total}}
    \end{aligned}
     \end{equation}
     \begin{equation}
    \begin{aligned}
        Micro\ Average\ Recall &= \frac{\sum_{i=1}^{n} \text{TP}_i}{\sum_{i=1}^{n} \text{Total Row}_i}
        &=\frac{\sum_{i=1}^{n} \text{TP}_i}{\text{Grand Total}}
    \end{aligned}
     \end{equation}
     \begin{equation}
    \begin{aligned}
        Micro\ Average\ F_1 =\frac{\sum_{i=1}^{n} \text{TP}_i}{\text{Grand Total}}
    \end{aligned}
     \end{equation}
     Da die Werte von Precision, Recall und F1-Score auf den aggregierten \ac{tp} basieren, sind sie im Micro Average identisch und spiegeln die Accuracy des Modells wider. Demgemäß wird bei dem \ac{cr} lediglich der Wert der Accuracy, wie in Tabelle \ref{tab:4.1} dargestellt, ausgewiesen. (Vgl. \citealt[S. 7-8]{Grandini:2020}).
    \item \textbf{\textit{Macro Average}}:\\
    Der Macro Average berechnet die Precision, den Recall und den F1-Score für jede Klasse \(i\) separat und bildet anschließend den Durchschnitt der einzelnen Werte.
     \begin{equation}
    \begin{aligned}
        Macro\ Average\ Precision = \frac{1}{n} \sum_{i=1}^{n} \text{Precision}_i
    \end{aligned}
     \end{equation}
     \begin{equation}
     \begin{aligned}
       Macro\ Average\ Recall = \frac{1}{n} \sum_{i=1}^{n} \text{Recall}_i
    \end{aligned}
     \end{equation}
     \begin{equation}
     \begin{aligned}
        Macro\ Average\ F_1 = \frac{1}{n} \sum_{i=1}^{n} \text{F}_1i
    \end{aligned}
     \end{equation}
   Der Macro Average gewichtet jede Klasse unabhängig von ihrer Größe gleich, wodurch die Leistung des Modells in allen Klassen gleichermaßen berücksichtigt wird. Dies ist insbesondere von Vorteil, wenn auch kleine Klassen mit wenigen Datenpunkten gleichwertig einbezogen werden sollen. (Vgl. \citealt[S. 6-7]{Grandini:2020}).
    \item \textbf{\textit{Weighted Average}}:\\
     Der Weighted Average berücksichtigt sowohl die Performance der einzelnen Klassen als auch den Support in jeder Klasse. In der Folge werden für jede Klasse Precision, Recall sowie F1-Score berechnet. Dabei findet eine Gewichtung nach dem Support pro Klasse.
     \begin{equation}
    \begin{aligned}
        Weighted\ Average\ Precision = \frac{\sum_{i=1}^{n} (\text{Precision}_i \cdot \text{Support}_i)}{\sum_{i=1}^{n} \text{Support}_i} 
    \end{aligned}
    \end{equation}
     \begin{equation}
     \begin{aligned}
       Weighted\ Average\ Recall = \frac{\sum_{i=1}^{n} (\text{Recall}_i \cdot \text{Support}_i)}{\sum_{i=1}^{n} \text{Support}_i} 
    \end{aligned}
    \end{equation}
     \begin{equation}
     \begin{aligned}
       Weighted\ Average\ F_1 = \frac{\sum_{i=1}^{n} (\text{F}_1i \cdot \text{Support}_i)}{\sum_{i=1}^{n} \text{Support}_i} 
    \end{aligned}
    \end{equation}
    Der Weighted Average berücksichtigt die Klassengröße und ist somit bei der Anwendung des Modells auf Datensätze mit unbalancierten Klassenverteilungen vorteilhaft. Dabei tragen Klassen mit einem höheren Support stärker zum Gesamtwert bei. (Vgl. \citealt[S. 68]{Tiwari:2022} sowie \citealt[S. 9]{Wang:2023}).
\end{itemize}

\subsection{Regularisierungs- \& Augmentierungstechniken}

Bei Aufgaben, bei denen große Datensätze nur schwer zu erlangen sind, ist das Risiko eines Overfittings des Netzwerks an den Trainingsdaten mit der Größe des Netzwerks signifikant erhöht. Zur Verhinderung vom Overfitting werden in der Praxis zwei gängige Ansätze verfolgt: Regularisierung und Datenaugmentation. (Vgl. \citealt[S. 31]{Lang:2023}).\\
Im Rahmen der Regularisierung wird die Anpassungsfähigkeit des Modells an die Trainingsdaten gezielt reduziert, was eine bessere Generalisierung auf unbekannte Daten zur Folge hat und zu einer verbesserten Leistung auf Validierungsdaten führen kann (vgl. \citealt[S. 146]{Chollet:2021}). Eine übliche Methode ist die Gewichtsregularisierung, bei der ein zusätzlicher Term zur Verlustfunktion hinzugefügt wird, der große Werte für die Modellparameter bestraft (vgl. \citealt[S. 148-149]{Chollet:2021}).Die L1-Regularisierung resultiert in der Regel in sparsamen Modellen, indem viele Parameter auf Null gesetzt werden, was dazu führt, dass das Modell nur die relevantesten Merkmale lernt (vgl. \citealt[S. 137-138]{Geron:2019}). Im Gegensatz dazu reduziert die L2-Regularisierung die Größe aller Parameter, wodurch übermäßig große Gewichtungen vermieden werden und das Modell weniger anfällig für Overfitting wird (vgl. \citealt[S. 135]{Geron:2019}). Dies impliziert eine Limitation der Modellwahl und eine Verhinderung der exakten Anpassung an die Trainingsdaten. Es lassen sich zwei Hauptarten unterscheiden:
\begin{enumerate}
    \item[] L1-Regularisierung: 
\begin{equation}
    \begin{aligned}
        Loss \rightarrow Loss + \lambda \sum_{j} |\theta_i|,
    \end{aligned} 
    \end{equation}
     wobei \(\theta_i\) die trainierbaren Netzparameter bezeichnet, während \(\lambda\) die Auswirkungen des Regularisierungsterms steuert.
     \item[] L2-Regularisierung: 
\begin{equation}
    \begin{aligned}
        Loss \rightarrow Loss + \lambda \sum_{j} \theta_i^2.
    \end{aligned} 
    \end{equation}     
\end{enumerate}
Die Entscheidung zwischen L1 und L2 hängt vom Problem ab. L1 erweist sich insbesondere bei sparsamen Modellen mit wenigen relevanten Parametern als vorteilhaft. Bei gleichmäßig verteilten Gewichten ist eine L2-Regulierung zu bevorzugen. (Vgl. \citealt[S. 138-139]{Geron:2019} sowie \citealt[S. 236]{Goodfellow:2017}). \\
Die Anwendung der Gewichtsregularisierung ist insbesondere bei kleinen \ac{dl}-Modellen von Nutzen. Bei großen \ac{dl}-Modellen, die häufig eine hohe Anzahl an Parametern aufweisen, zeigt sich, dass die Einführung von Beschränkungen für Gewichtswerte durch Gewichtsregularisierung einen vergleichsweise geringen Einfluss auf die Modellkapazität hat. In diesen Fällen werden daher andere Regularisierungstechniken wie Dropout eingesetzt. (Vgl. \citealt[S. 150]{Chollet:2021}).\\
Im Rahmen des Dropouts werden Neuronen zufällig ausgewählt, was deren Ausschluss aus dem Optimierungsprozess zur Folge hat. Dies resultiert in einer Reduktion der Speicherkapazität einzelner Neuronen für spezifische Datenpunkte, wodurch ein Overfitting verhindert wird. Der Vorteil von Dropout besteht in der Induktion eines Ensemble-Effekts, da während des Trainings verschiedene Teilmodelle mit zufälligen neuronalen Konfigurationen generiert werden, was die Generalisierungsfähigkeit erhöht. (Vgl. \citealt[S. 365-367]{Geron:2019}).
\\
Die Datenaugmentation bezeichnet eine Methode der künstlichen Vergrößerung von Datensätzen, welche durch Modifikationen der Trainingsdaten erfolgt. In der Bildverarbeitung finden Techniken wie zufällige Rotationen, Spiegelungen, Zoom, Zuschneiden, Hinzufügen von Rauschen sowie Anpassungen von Helligkeit, Kontrast und Sättigung Anwendung. Die genannten Transformationen erhöhen die Robustheit des Modells, da dieses lernt, zugrundeliegende Muster in Bildern unabhängig von den genannten Variationen zu erkennen. Dies führt nicht nur zu einer Reduktion des Risikos eines Overfittings, sondern auch zu einer Verbesserung der allgemeinen Leistungsfähigkeit des Modells. (Vgl. \citealt[S. 465]{Geron:2019}).\\
Insbesondere in Szenarien, die durch limitierte und unausgewogene Datensätze gekennzeichnet sind, erweist sich die Datenaugmentation als vorteilhaft. Die Anwendung von Datenaugmentation ermöglicht es Modellen, Merkmale aus unterschiedlichen Perspektiven und unter variierenden Lichtbedingungen besser zu erkennen, was zu einer Erhöhung der Klassifikationsgenauigkeit beiträgt. Des Weiteren unterstützt sie das Erlernen algorithmischer Vorannahmen über Bildinvarianten, was insbesondere bei der Analyse medizinischer Bilddaten wichtig sein könnte. (Vgl. \citealt[S. 31-32; 48]{Lang:2023}).\\
In der medizinischen Bildverarbeitung sind mit der Datenaugmentation jedoch spezifische Herausforderungen verbunden. Transformationsprozesse wie Rotation oder Zuschneiden können die diagnostische Aussagekraft von Daten beeinträchtigen. Dies kann durch eine fehlerhafte Labelzuweisung oder eine Veränderung kritischer Details bedingt sein. Diese Modifikationen führen zu einer Einschränkung der klinischen Anwendbarkeit der augmentierten Daten. Des Weiteren weisen medizinische Bilder häufig eine höhere Variabilität auf als lediglich geometrische und helligkeitsbezogene Unterschiede. Dies begrenzt die Effektivität herkömmlicher Augmentationstechniken. Fortgeschrittene Ansätze wie Generative Adversarial Networks (GANs) eröffnen vielversprechende Möglichkeiten zur Generierung synthetischer Daten. Allerdings ist zu berücksichtigen, dass GANs eine hohe Anzahl an Ausgangsdaten sowie einen umfassenden Modellierungsaufwand erfordern. Zudem zeigt sich in der Praxis, dass die von GANs erzeugte Bildqualität häufig nicht den Anforderungen klinischer Anwendungen genügt. (Vgl. \citealt[S. 9; 27]{Shorten:2019}).

\subsection{Konvolutionale neuronale Netze}

Konvoultionale neuronale Netze (engl. \textit{\acf{cnn}}) stellen eine bedeutende Architektur von \ac{dl}-Modellen dar, die insbesondere für die Verarbeitung und Analyse von Bilddaten zum Einsatz kommt (vgl. \citealt[S. 445]{Geron:2019}). Im Gegensatz zu traditionellen \ac{mlp} ist die Verarbeitung von Daten durch \ac{cnn} in ihrem natürlichen zweidimensionalen (2D) Format möglich, wodurch die räumlichen Beziehungen zwischen den Pixeln erhalten bleiben (vgl. \citealt[S. 448]{Geron:2019}). Im Falle von \ac{mlp} erfolgt eine Umwandlung der Eingabebilder in einen eindimensionalen (1D) Vektor, was zu einer signifikanten Vergrößerung der Eingabegrößen und einer Vielzahl trainierbarer Parameter führt (vgl. \citealt[S. 447-448]{Geron:2019}). Diese Vorgehensweise erweist sich für die Verarbeitung von Bildern jedoch als suboptimal, da \ac{mlp}s keine Translationsinvarianz aufweisen, was bedeutet, dass Muster nicht erkannt werden können, wenn sie sich an verschiedenen Positionen im Bild befinden (vgl. \citealt[S. 203]{Chollet:2021}). \\
\ac{cnn}s beheben diese Probleme durch die Anwendung von Konvolutionsoperationen (engl. \textit{convolutional operations}) , welche von der Struktur des visuellen Kortex im Gehirn inspiriert sind. Die Studien von \citet[]{Hubel:1959}, \citet[]{Hubel:1959_} und \citet[]{Hubel:1968} führten zu der Erkenntnis,  dass der visuelle Kortex aus Schichten mit hierarchischer Struktur besteht. Eine Vielzahl von Neuronen im visuellen Kortex weist ein kleines lokales rezeptives Feld (engl. \textit{local receptive field}) auf, welches lediglich auf visuelle Stimuli in einem begrenzten Bereich des Gesichtsfelds reagiert. In den frühen Schichten werden einfache Merkmale wie Kanten erkannt, während die tieferen Schichten diese zu komplexeren Formen und letztlich zu vollständigen Objekten kombinieren. Die genannten Erkenntnisse führten zunächst zur Anwendung von manuell erstellten Filtern zur Merkmalsextraktion aus Bilddaten und schließlich zur Entwicklung von \ac{cnn}s. (Vgl. \citealt[S. 446-447]{Geron:2019}).\\
Konvolutionale Schichten (engl. \textit{convolutional layers}) stellen die zentralen Bausteine eines \ac{cnn}s dar. Im Gegensatz zu vollständig verbundenen Schichten weisen Neuronen in den konvolutionalen Schichten lediglich eine geringe Verbindungsstärke zu einem kleinen Bereich der Eingabe auf, dem sogenannten rezeptiven Feld. Infolgedessen können in den initialen Schichten einfache Charakteristika wie Kanten identifiziert und in den nachfolgenden Schichten zu komplexeren Mustern kombiniert werden. Dies trägt maßgeblich zur hohen Effizienz von \ac{cnn}s bei der Bearbeitung von Bilddaten bei, da die visuelle Welt prinzipiell räumlich hierarchisch ist. Um die Schichtgröße konstant zu halten, wird häufig Zero Padding angewendet, bei dem Nullen an den Eingaberändern hinzugefügt werden. Der Stride gibt an, wie stark sich das rezeptive Feld von einem Schritt zum nächsten verschiebt, und ermöglicht es, die Modellkomplexität zu verringern, indem große Eingaben zu kleineren Ausgabeschichten verbunden werden. (Vgl. \citealt[S. 448-449]{Geron:2019}).\\
Ein Filter, auch Konvolutionskern genannt, besteht aus den Gewichtungen eines Neurons und kann als kleines Bild in der Größe des rezeptiven Feldes dargestellt werden. Ein Beispiel wäre ein 3×3-Filter, bei dem nur die mittlere vertikale Linie mit Einsen gefüllt ist, während der Rest Nullen enthält. Dieser Filter aktiviert nur das zentrale vertikale Merkmal in einem Bild, während andere Bereiche des Bildes ignoriert werden. Ein anderer Filter könnte in ähnlicher Weise nur horizontale Linien erkennen. Wenn alle Neuronen in einer Schicht den gleichen Filter verwenden, erzeugen sie eine Merkmalskarte (engl. \textit{feature map}), die die aktivierten Bildbereiche hervorhebt. Während des Trainings lernen Konvolutionale Schichten automatisch, welche Filter am besten geeignet sind, um bestimmte Merkmale in einem Bild zu erkennen, und kombinieren diese in den oberen Schichten zu komplexeren Mustern. Dieser Lernprozess ermöglicht es \ac{cnn}s, immer abstraktere Darstellungen der Bilddaten zu entwickeln. (Vgl. \citealt[S. 450]{Geron:2019}).\\
Pooling-Schichten (engl. \textit{pooling layer}) in \ac{cnn}s weisen eine ähnliche Funktionsweise wie konvolutionale Schichten auf. Auch hier ist jedes Neuron mit einer begrenzten Anzahl von Neuronen der vorherigen Schicht verbunden, wobei diese in einem kleinen rezeptiven Feld liegen. Im Gegensatz zu konvolutionalen Schichten verfügen Pooling-Neuronen nicht über Gewichte, sondern aggregieren die Eingaben durch eine Aggregationsfunktion wie beispielsweise den Max- oder Mittelwert. In der Praxis findet vornehmlich das Max-Pooling Anwendung, bei dem lediglich der jeweils höchste Wert aus jedem rezeptiven Feld an die nachfolgende Schicht weitergeleitet wird, während die übrigen Werte verworfen werden. (Vgl. \citealt[S. 457]{Geron:2019}).\\ 
Die Verwendung von Max-Pooling-Schichten führt zu einer Reduktion der Rechenzeit, des Speicherbedarfs sowie der Anzahl der Parameter. Zudem wird eine gewisse Invarianz gegenüber kleinen Verschiebungen eingeführt. Dadurch wird das Modell weniger anfällig für kleine Verschiebungen in den Bilddaten, was die Generalisierungsfähigkeit erhöht. Zudem bietet das Max-Pooling eine begrenzte Rotations- und Skalierungsinvarianz, was insbesondere bei Klassifikationsaufgaben vorteilhaft ist, bei denen die Vorhersagen von diesen Details unabhängig sein sollen. (Vgl. \citealt[S. 457-458]{Geron:2019}).\\
Die Architektur von \ac{cnn}s umfasst in der Regel mehrere konvolutionale Schichten (+ ReLUs) und Pooling-Schichten, gefolgt von vollständig verbundenen Schichten, welche die extrahierten Merkmale verarbeiten und letztlich zur Klassifikation genutzt werden. Abbildung \ref{fig:4.7} zeigt die Architekur eines \ac{cnn}s. (Vgl. \citealt[S. 460-461]{Geron:2019}).
\begin{figure}[H]
    \centering 
    \includegraphics[width=0.9\textwidth]{BA/Abbildungen/CNN Architektur.jpg} % Bild einfügen, anpassen der Breite
    \caption{Architektur eines \ac{cnn}s, entnommen aus \citet[S. 461]{Geron:2019}.} % Beschriftung der Abbildung
    \label{fig:4.7} % Label für Referenz
\end{figure}
Zusammenfassend lässt sich festhalten, dass die Verarbeitung dreidimensionaler Daten durch \ac{cnn}s möglich ist, insbesondere im Kontext medizinischer Bildgebungsverfahren wie \ac{mrt}- oder \ac{ct}-Scans. In diesen Fällen erfolgt nicht lediglich eine Stapelung der Merkmalskarten in zwei Dimensionen (Höhe und Breite), sondern ebenfalls in der dritten Dimension, wodurch die räumlichen Informationen eines Volumendatensatzes berücksichtigt werden. Die gleichzeitige Verwendung mehrerer Merkmalskarten erlaubt die Extraktion verschiedener Aspekte eines Bildvolumens. Dies ermöglicht es \ac{cnn}s, komplexe 3D-Strukturen zu erkennen und wichtige Muster in dreidimensionalen Daten an unterschiedlichen Positionen mit hoher Effizienz zu verarbeiten. Diese Netze sind speziell darauf ausgelegt, Volumendaten in verschiedenen Schichten hierarchisch zu verarbeiten und zu abstrahieren. Diese Eigenschaft erweist sich insbesondere in der medizinischen Bildgebung als vorteilhaft, da dort eine genaue Erkennung und Segmentierung von Strukturen über mehrere Schichten hinweg erforderlich ist. (Vgl. \citealt[S. 451]{Geron:2019} sowie \citealt[S. 35]{Lang:2023}).

\subsection{Segmentierungsnetze}

\ac{dl}-Modelle finden neben der Klassifikation auch bei der Bildsegmentierung Anwendung. Es existieren zwei Hauptkategorien der Segmentierung: die semantische Segmentierung (engl. \textit{semantic segmentation}) und die Instanzsegmentierung (engl. \textit{instance segmentation}). Im Rahmen der semantischen Segmentierung erfolgt eine pixelweise Klassifikation von Bildbereichen in vordefinierte Kategorien, wobei eine weitere Differenzierung von Objekten derselben Klasse nicht stattfindet. Im Gegensatz dazu erfolgt bei der Instanzsegmentierung eine separate Segmentierung verschiedener Objekte derselben Klasse. (Vgl. \citealt[S. 231-232]{Chollet:2021}). 
Bezüglich der Segmentierung von \ac{mrt}-Bildern des Gehirns bedeutet dies, dass bei der semantischen Segmentierung die Aufgabe darin besteht, Tumorgewebe von gesundem Gewebe zu unterscheiden, während bei der Instanzsegmentierung eine Differenzierung zwischen verschiedenen Instanzen desselben Tumors theoretisch möglich wäre.\\
Eine der Hauptschwierigkeiten bei der Segmentierung mit herkömmlichen \ac{cnn}s besteht darin, dass die räumliche Auflösung der Bilder im Laufe der Verarbeitung schrittweise verloren geht. Dies hat zur Konsequenz, dass das Modell am Ende lediglich über eine unzureichende Vorstellung hinsichtlich der Lage eines Objekts verfügt, ohne genauere Informationen zu besitzen. Zur Lösung dieser Problematik wurden verschiedene Ansätze entwickelt, darunter Architekturen wie Autoencoder, die speziell für Segmentierungsaufgaben angepasst sind. (Vgl. \citealt[S. 492]{Geron:2019}). \\
Die U-Net-Architektur basiert auf einem Autoencoder-Ansatz, welcher aus einem Encoder und einem Decoder besteht wie in Abbildung \ref{fig:4.8} dargestellt. Dabei wird das Bild durch den Encoder in eine latente Repräsentation transformiert, welche durch den Decoder wiederum in die ursprüngliche Auflösung zurückgeführt wird. (Vgl. \citealt[S. 569]{Geron:2019}). Der Einsatz von transponierten konvolutionellen Schichten (engl. \textit{transpose convolutions}) ermöglicht die Wiederherstellung der räumlichen Dimensionalität im Decoder (vgl. \citealt[S. 493]{Geron:2019}). Ein wesentliches Charakteristikum des U-Net ist die Verwendung von sogenannten Skip-Verbindungen (engl. \textit{skip connections}), welche eine unmittelbare Weitergabe von Informationen aus den initialen Schichten des Encoders an die korrespondierenden Schichten des Decoders ermöglichen (vgl. \citealt[S. 494]{Geron:2019}). Dies gewährleistet die Beibehaltung der frühen, globalen Bildmerkmale über das gesamte Netzwerk, was zu einer präzisen Segmentierung führt (vgl. \citealt[S. 255]{Deprez:2023}).
\begin{figure}[ht]
    \centering 
    \includegraphics[height=0.6\textwidth]{BA/Abbildungen/U-Net Architektur.jpg} % Bild einfügen, anpassen der Breite
    \caption{U-Net-Architektur, entnommen aus \citet[S. 2]{Ronneberger:2015}. Die Bezeichnungen "`\textit{copy and crop}"' beziehen sich auf Skip-Verbindungen, während "`\textit{up-conv}"' für transponierte konvolutionelle Schichten steht.} % Beschriftung der Abbildung
    \label{fig:4.8} % Label für Referenz
\end{figure}\\
Im Rahmen der Segmentierungsaufgabe findet in der Praxis üblicherweise eine Bewertung der Übereinstimmung zwischen den vorhergesagten und den tatsächlichen Segmentierungen unter Verwendung des Dice-Koeffizienten statt. Der Dice-Koeffizient wird gemäß folgender Definition ermittelt:
\begin{equation}
    DSC(A, B) = \frac{2 |A \cap B|}{|A| + |B|},
\end{equation}
wobei \(A\) und \(B\) die Menge der Punkte in den Segmentierungen darstellen. Ein Dice-Wert von 1 impliziert eine perfekte Übereinstimmung, während 0 keine Übereinstimmung anzeigt. (Vgl. \citealt[S. 37]{Lang:2023}).\\
Der zu minimierende Dice-Verlust lässt sich wie folgt berechnen: 
\begin{equation}
    L_{DSC} = 1 - DSC(A, B).
\end{equation}
 Zur Messung der maximalen Abweichung zwischen den Segmentierungen wird neben dem Dice-Koeffizienten auch der Hausdorff-Abstand verwendet. Der Hausdorff-Abstand wird wie folgt definiert:
 \begin{equation}
    HD(A, B) = \max\left( \sup_{a \in A} \inf_{b \in B} d(a, b), \sup_{b \in B} \inf_{a \in A} d(a, b) \right)
\end{equation}
wobei \( d(a, b) \) die euklidische Distanz zwischen den Punkten \( a \) und \( b \) darstellt und \( \sup \) sowie \( \inf \) das Supremum und das Infimum sind. (Vgl. \citealt[S. 37]{Lang:2023}).\\
Der Einsatz solcher Architekturen und Verlustfunktionen ermöglicht es \ac{cnn}s, zunehmend komplexere Bilddaten zu analysieren und präzise Segmentierungen durchzuführen.

\subsection{Transfer Learning}

Die zentrale Idee des \acf{tl} besteht in der Nutzung von Wissen, welches in einem spezifischen Kontext [A] erlernt wurde, zu nutzen, um die Generalisierungsleistung in einem anderen Kontext [B] zu verbessern (vgl. \citealt[S. 536]{Goodfellow:2017}). Dies ist besonders relevant, wenn für die Modellierung nicht genügend umfangreiche Datensätze zur Verfügung stehen (vgl. \citealt[S. 218-219]{Chollet:2021}). Modelle, die auf umfangreichen Datensätzen trainiert wurden, erlernen grundlegende visuelle Charakteristika, die auf andere Domänen transferierbar sind (vgl. \citealt[S. 218-219]{Chollet:2021}). Dies impliziert, dass Merkmale, die in [A] erfasst wurden, auch in [B] nützlich sein können, da zahlreiche visuelle Informationen, wie Kanten und Formen, domänenübergreifend von Bedeutung bleiben (vgl. \citealt[S. 219-220]{Chollet:2021}). \\
Innerhalb der medizinischen Bildverarbeitung, in der häufig lediglich kleine Datensätze zur Verfügung stehen, hat sich das \ac{tl} als besonders wertvoll erwiesen. Ein Modell, welches beispielsweise auf dem ImageNet-Datensatz mit Millionen von Bildern aus verschiedenen Alltagsszenarien trainiert wurde, kann auch für medizinische Bilddaten, wie beispielsweise \ac{mrt}-Scans, eingesetzt werden. Obwohl die Domänen – natürliche Bilder und medizinische Bilder – signifikante Differenzen aufweisen, erweisen sich grundlegende visuelle Charakteristika, die auf ImageNet erlernt wurden, auch für die medizinische Bildverarbeitung als nützlich. (Vgl. \citealt[S. 39]{Lang:2023}). \\
Im Rahmen der Anwendung von \ac{tl} erfolgt in der Regel eine Übertragung der Gewichtungen aus einem vortrainierten Modell auf das neue Modell. Diesbezüglich sind insbesondere die initialen Schichten des Netzwerks von Interesse, da sie für die Erkennung einfacher Merkmale verantwortlich sind. Für spezifische Aufgabenstellungen, wie beispielsweise die Erkennung von Tumoren in medizinischen Bilddaten, werden die letzten Schichten des Netzwerks modifiziert oder ersetzt. Dieser Prozess wird als \acf{ft} bezeichnet und beinhaltet eine erneute Trainingsphase, um das Modell optimal an die jeweilige Problemstellung anzupassen. (Vgl. \citealt[S. 345-347]{Geron:2019}). \\
Die Verwendung medizinischer Bilddaten, wie \ac{mrt}-Scans, erfordert oft die Verarbeitung von 3D-Daten. Dies stellt eine Herausforderung dar, da vortrainierte Modelle in der Regel auf die Analyse von 2D-Daten ausgelegt sind. Um dennoch 3D-Daten nutzen zu können, werden häufig 2D-Schnitte in die Farbkanäle des Modells eingefügt, was jedoch mit einem Verlust an räumlicher Information einhergeht. Eine bessere Lösung besteht in der Entwicklung speziell für 3D-Daten optimierter \ac{tl}-Modelle. Studien zur medizinischen Bildverarbeitung zeigen unterschiedliche Ergebnisse. Einige konnten keine signifikanten Vorteile durch \ac{tl} nachweisen, während andere Studien klare Verbesserungen feststellten. Daher sollte die Anwendung von \ac{tl} sorgfältig im Hinblick auf die konkrete Aufgabe und den verfügbaren Datensatz abgewogen werden. (Vgl. \citealt[S. 48]{Lang:2023}). 
\newpage

\section{Datensatz}

\subsection{Beschreibung der Daten}

Für diese Arbeit wurde ein offen zugänglicher \ac{mrt}-Datensatz aus dem Kaggle-Repository verwendet \citet[]{Nickparvar:2021}. Der Datensatz umfasst drei öffentlich zugängliche Quellen: Figshare \citet[]{Cheng:2017} , SARTAJ \citet[]{Bhuvaji:2020}, und BR35H \citet[]{Hamada:2017}. In diesem Zusammenhang werden verschiedene \ac{mrt}-Bilder des menschlichen Gehirns präsentiert, welche eine Vielfalt an \ac{ht} und deren charakteristische Eigenschaften repräsentieren.\\
Der Datensatz enthält insgesamt 7.023 RGB-Bilder mit einer Auflösung von 512 x 512 Pixeln. Diese umfassen vier Klassen:
\begin{itemize}
    \item Gliome (engl. \textit{glioma}): 1.621 Bilder
    \item Meningeome (engl. \textit{meningioma}): 1.645 Bilder
    \item Hypophysentumoren (engl. \textit{pituitary}): 1.757 Bilder
    \item Nicht tumoröse Gewebe (engl. \textit{no tumor}): 2.000 Bilder
\end{itemize}
Diese Verteilung gewährleistet eine nahezu ausgewogene Repräsentation der häufigsten Tumorarten, was für das Training von Klassifikationsmodellen von essenzieller Bedeutung ist. Dennoch lässt sich eine leichte Tendenz zugunsten der nicht-tumorösen Bilder beobachten, die bei der Modellentwicklung zu berücksichtigen ist, um Verzerrungen zu minimieren. \\
Die Größe und Diversität des Datensatzes ermöglichen eine robuste Modellierung und ermöglichen die Erkennung generalisierbarer Muster. Abbildung \ref{fig:5.1} stellt die Klassenverteilung grafisch dar. 
\begin{figure}[ht]
    \centering
    \hspace{-1.0cm} % Verschiebt die Abbildung nach links
        \includegraphics[width=0.7\textwidth]{BA/Abbildungen/Klassenverteilung im Datensatz.png}
        \caption{Verteilung der Bildklassen im Datensatz.}
    \label{fig:5.1}
\end{figure}
Die Aufteilung des Datensatzes erfolgte in 80\% Trainingsdaten und 20\% Testingdaten, wobei letztere nochmals in 30\% Validierung und 70\% Testing unterteilt wurden. Diese Verteilung gewährleistet eine repräsentative Mischung für das Training und die Evaluierung der Modelle. Eine grafische Darstellung ist in Abbildung \ref{fig:5.2} zu sehen.
\begin{figure}[ht]
    \centering
        \includegraphics[width=0.54\textwidth]{BA/Abbildungen/Struktur der Datenaufteilung.png}
        \caption{Verteilung der Daten auf Trainings-, Validierungs- und Testsets.}
    \label{fig:5.2}
\end{figure}

\subsection{Datenvorverarbeitung}

Um eine konsistente Eingabegröße für das Modell zu gewährleisten, wurden die RGB-Bilder von ihrer ursprünglichen Auflösung von 512 x 512 Pixeln auf 224 x 224 Pixel standardisiert. Diese Anpassung der Bildgröße ist erforderlich, um die Kompatibilität der Bildgröße zwischen dem selbst entwickelten Modell und dem vortrainierten Modell ResNet-50 sicherzustellen (vgl. \citealt[S. 4]{He:2016} sowie \citealt[S. 346]{Geron:2019}). Da es sich um RGB-Bilder handelt, wurden die drei Farbkanäle für jeden Pixel beibehalten und die Bilddaten unter Verwendung des Parameters \texttt{\colorbox{gray!20}{scale=1/255}} normalisiert. Dies führte dazu, dass die Pixelwerte für jeden Farbkanal von ihrem ursprünglichen Bereich (0–255) auf einen Bereich von 0–1 skaliert wurden. (Vgl.  \citealt[S. 453]{Goodfellow:2017}). \\
Des Weiteren wurde durch den Einsatz des Keras-Generators \texttt{\colorbox{gray!20}{ImageDataGenerator}} eine effiziente Pipeline zur Aufteilung der Daten in Testing- und Validierungsset implementiert, wobei der Parameter \texttt{\colorbox{gray!20}{validation\_split}} verwendet wurde. Dies gewährleistet, dass die Modelle während des Trainings optimal vorbereitet werden (vgl. \citealt[S. 482]{Geron:2019}). Die Kombination aus Standardisierung der Auflösung und Normalisierung der Werte ermöglicht es dem Modell, Eingabedaten in einem einheitlichen Format zu verarbeiten, was die Trainingsstabilität und Modellleistung fördert. \\
Zusätzlich wurde zur Gewährleistung einer reproduzierbaren Durchführung die Steuerung zufälliger Operationen durch das Setzen eines globalen Seeds vorgenommen. Dazu wurden die Seeds in sämtlichen relevanten Modulen, beispielsweise NumPy und TensorFlow, einheitlich definiert. Dies gewährleistet, dass die Datenaufteilung sowie weitere zufällige Operationen deterministisch erfolgt und die Ergebnisse von Experimenten reproduzierbar bleiben.\\
Da der Datensatz nahezu ausgewogen ist, wurde auf die Anwendung von Datenaugmentation verzichtet, um sicherzustellen, dass keine relevanten Bildmerkmale verloren gehen, die für die Tumorklassifikation von entscheidender Bedeutung sind, und um die klinische Anwendbarkeit der Ergebnisse nicht einzuschränken.\\
Obwohl die Klassenverteilung im Datensatz nahezu ausgewogen ist, wurde ein Gewichtungsschema implementiert, um die Modellleistung zu optimieren (vgl. \citealt[S. 304]{Geron:2019}). Die Funktion \texttt{\colorbox{gray!20}{compute\_class\_weight}} wurde eingesetzt, um Klassengewichte zu berechnen, wodurch eine mögliche Verzerrung durch die leicht höhere Anzahl an nicht tumorösen Bildern verhindert werden sollte. Diese Maßnahme resultierte in einer Verbesserung der Modellleistung über alle Klassen hinweg.

\subsection{Limitation des  Datensatzes}

Obschon der Datensatz eine umfassende Auswahl an \ac{ht} umfasst, bestehen Einschränkungen hinsichtlich der Variabilität im Vergleich zur klinischen Realität. Die in öffentlichen Datensätzen enthaltenen Informationen decken nicht immer die gesamte Bandbreite seltener Tumorarten ab, was die Generalisierbarkeit der Modelle auf klinische Daten aus der Praxis beeinträchtigen könnte. Eine potenzielle Erweiterung könnte in der Anwendung der entwickelten Modelle auf klinische, nicht-öffentliche Datensätze liegen. Dadurch könnten die Robustheit und Übertragbarkeit der Modelle überprüft werden. 
\newpage

\section{Implementierung der Modelle}

\subsection{Betriebssystem \& Software-Pakete}

Für die Implementierung und das Training der Modelle wurde eine speziell auf die Anforderungen moderner \ac{dl}-Anwendungen abgestimmte Hardware- und Softwareumgebung verwendet. Im Folgenden wird das verwendete Betriebssystem, die Hardware sowie die eingesetzten Software-Pakete und deren Relevanz für das Projekt beschrieben.
\subsubsection{Betriebssystem und Hardware}
Die Implementierung erfolgte unter Verwendung des Betriebssystems Windows 11 Pro. Als Hardware stand ein System mit folgenden Spezifikationen zur Verfügung:
\begin{itemize}
    \item Prozessor (CPU): Intel Core i9
    \item Arbeitsspeicher (RAM): 32 GB
    \item Grafikkarte (GPU): NVIDIA RTX 4070 Laptop GPU mit 8 GB Video-RAM
\end{itemize}
Die Nutzung der NVIDIA GPU ermöglichte eine signifikante Beschleunigung des Trainingsprozesses, da sowohl TensorFlow als auch Keras eine Optimierung für 
GPU-beschleu-nigtes Rechnen bieten. Die leistungsstarke CPU, der ausreichende Arbeitsspeicher und die moderne GPU erwiesen sich als essenziell für die effiziente Verarbeitung der umfangreichen Bilddaten und die komplexen Berechnungen der Modelle.
\subsubsection{Software-Umgebung}
Die Implementierung der Modelle erfolgte in der Programmiersprache Python 3.9.13 unter Verwendung der Entwicklungsumgebung Jupyter Notebook. Jupyter ermöglichte die nahtlose Integration von Code, Visualisierungen und begleitender Dokumentation.
Im Rahmen der Implementierung wurden folgende Python-Bibliotheken und Pakete eingesetzt:
\begin{itemize}
    \item Allgemeine Bibliotheken
    \begin{itemize}
    \item TensorFlow von \citet[]{Tensorflow:2015}: Framework für \ac{dl}-Anwendungen, das insbesondere für die Modellarchitektur, das Training und die Optimierung der Modelle genutzt wurde.
    \item Keras von \citet[]{Keras:2015}: High-Level-API von TensorFlow, die den Aufbau und die Anpassung neuronaler Netze erleichtert.
    \item sklearn.utils (\texttt{\colorbox{gray!20}{shuffle}}): Zum Durchmischen der Daten, um eine gleichmäßige Verteilung der Klassen im Training zu gewährleisten.
    \item NumPy von \citet[]{Harris:2020}: Zur effizienten Verarbeitung numerischer Daten.
    \item pandas von \citet[]{Mckinney:2010}: Für die Analyse und Manipulation tabellarischer Daten, insbesondere durch die Verwendung der DataFrame-Struktur. 
    \item os und random: Für die Verwaltung des Dateisystems und die Reproduzierbarkeit von Experimenten.
    \item collections (\texttt{\colorbox{gray!20}{Counter}}): Zur statistischen Analyse und Darstellung der Häufigkeit von Datenelementen. 
    \item warnings: Zur Unterdrückung irrelevanter Warnmeldungen.
    \end{itemize}
    \item Visualisierungsbibliotheken
    \begin{itemize}
    \item Matplotlib von \citet[]{Hunter:2007} und Seaborn von \citet[]{Waskom:2021}: Für die Visualisierung von Datenverteilungen, Modellmetriken und Trainingsergebnissen.
    \item sklearn.metrics von \citet[]{Pedregosa:2011}: Zur Berechnung von Leistungsmetriken für Modellbewertung.
    \item visualkeras: Zur grafischen Darstellung der Modellarchitektur.
    \end{itemize}
    \item Datenvorbereitung
    \begin{itemize}
    \item sklearn.utils (\texttt{\colorbox{gray!20}{class\_weight}}): Zum Ausgleich unausgeglichener Klassenverteilungen.
    \item ImageDataGenerator: Zur Skalierung der Daten und Unterteilung des Datensatzes in Trainings- und Validierungs-Subsets unter Verwendung eines festen Seeds für Reproduzierbarkeit.
    \end{itemize}
    \item Modellbildung und Training mit TensorFlow/Keras 
    \begin{itemize}
    \item Modellarchitektur: Aufbau der \ac{cnn}s mit Schichten wie \texttt{\colorbox{gray!20}{Conv2D}}, \texttt{\colorbox{gray!20}{Dense}},  \texttt{\colorbox{gray!20}{Dropout}}, \texttt{\colorbox{gray!20}{MaxPooling2D}}, \texttt{\colorbox{gray!20}{GlobalAveragePooling2D}}, \texttt{\colorbox{gray!20}{BatchNormalization}} und \texttt{\colorbox{gray!20}{Flatten}}.
    \item Training: Implementierung von Optimierungstechniken wie \texttt{\colorbox{gray!20}{Adam}} sowie Regularisierung mit \texttt{\colorbox{gray!20}{l2}} und Verwendung von Verlustsfunktionen wie \ac{cce} \\ \texttt{\colorbox{gray!20}{categorical\_crossentropy}}.
    \item Keras Tuner (kt): Für das \acf{hp}-Tuning.
    \item Callback (\texttt{\colorbox{gray!20}{EarlyStopping}}: Zur Optimierung des Trainingsprozesses und zur Vermeidung von Overfitting.
    \end{itemize}
\end{itemize}
Die gewählte Softwareumgebung sowie die verwendeten Bibliotheken wurden sorgfältig dokumentiert, um die Nachvollziehbarkeit und Reproduzierbarkeit der Ergebnisse zu gewährleisten. Die spezifischen Versionen der wesentlichen Pakete sind in Tabelle \ref{tab:5.1} zusammengefasst.

\begin{table}[ht]
\centering
\begin{tabular}{||c|c||}
\hline
\textbf{Bibliothek} & \textbf{Version} \\
\hline
TensorFlow & 2.10.1 \\
Keras & 2.10.0 \\
NumPy & 1.21.5 \\
pandas & 1.4.4 \\
Matplotlib & 3.5.2 \\
Seaborn & 0.11.2 \\
scikit-learn & 1.0.2 \\
\hline
\end{tabular}
\caption{Softwarebibliotheken und deren spezifische Versionen.}
\label{tab:5.1}
\end{table}

Die Kombination der genannten Software-Pakete ermöglichte eine effiziente und flexible Implementierung der Modelle. Die Verwendung von TensorFlow und Keras als robuste Plattform für die Modellbildung und das Training erwies sich als vorteilhaft, ebenso wie die Integration von Visualisierungsbibliotheken wie Matplotlib und Seaborn, welche eine umfassende Analyse und Präsentation der Ergebnisse erleichterten.
Die Hardware- und Softwarekonfiguration bildete somit eine optimale Grundlage für die Implementierung der Modelle und die Verarbeitung des \ac{mrt}-Datensatzes.


\subsection{Selbst entwickeltes CNN-Modell}
\subsubsection{Modellarchitektur}

Das entwickelte \ac{cnn} wurde mit dem Ziel entworfen, die Klassifikation von 
\ac{mrt}-Bildern in vier Klassen zu ermöglichen. Die Architektur des Modells ist in Abbildung \ref{fig:6.2} dargestellt und beinhaltet sowohl konvolutionale Schichten zur Merkmalsextraktion als auch dichte Schichten zur finalen Klassifikation.
\begin{figure}[ht]
    \centering
        \includegraphics[width=.93\textwidth]{BA/Abbildungen/Selbsterstelltes CNN Architektur.jpg}
        \caption{Architektur vom selbst entwickelten \ac{cnn}-Modell.}
    \label{fig:6.2}
\end{figure}

Die Architektur des Modells umfasst zunächst eine Eingabeschicht und eine Abfolge von vier konvolutionalen Blöcken, welche jeweils eine Kombination aus einer konvolutionale Schicht, einer Batch-Normalisierung und einer Max-Pooling-Schicht enthalten. Im ersten Block werden 32 Filter mit einer Kernelgröße von 4×4 verwendet, wobei die ReLU-Aktivierungsfunktion für die Einbindung von Nichtlinearität sorgt. In den nachfolgenden Blöcken erfolgt eine sukzessive Erhöhung der Filteranzahl (64, 128 sowie 256), um komplexere Merkmale zu extrahieren und somit die Repräsentationsfähigkeit des Modells zu verbessern.
\\
Die Batch-Normalisierung, welche in jedem Block nach der konvolutionalen Schicht implementiert wird, dient der Stabilisierung der Eingabeverteilung für jede Schicht. Dies resultiert in einer Beschleunigung des Trainings sowie einer Minimierung des Risikos der internen Kovariatenverschiebung (engl. \textit{internal covariate shift}), bei dem sich die Verteilung der Eingabedaten für eine Schicht während des Trainings verändert. (Vgl. \citealt[S. 338-341]{Geron:2019}).
\\
\\
Die Max-Pooling-Schichten mit einer Poolgröße von 3×3 reduzieren die räumliche Dimension der Merkmalskarten und tragen zur Verringerung der Rechenkomplexität sowie zur Erhöhung der translationalen Invarianz bei. In der Folge der Merkmalsextraktion erfolgt durch eine Flatten-Schicht die Überführung in einen eindimensionalen Vektor. Dieser bildet die Grundlage für die nachfolgende Klassifikation.
\\
Im dichten Teil des Netzwerks folgt eine vollverbundene Schicht mit 1024 Neuronen. Im Rahmen dieser Konfiguration findet eine L2-Regularisierung mit $\lambda = 0,001$ Anwendung, um ein Overfitting des Modells zu vermeiden. Des Weiteren wird durch den Einsatz von Dropout-Schicht (50\%) die Generalisierungsfähigkeit des Modells optimiert, indem während des Trainings zufällig Neuronen deaktiviert werden. Die Architektur endet mit einer Ausgabeschicht, die vier Neuronen enthält, was der Anzahl der zu unterscheidenden Klassen entspricht, sowie einer Softmax-Aktivierungsfunktion, welche die Wahrscheinlichkeiten der einzelnen Klassen berechnet.
\\
Das Training des Modells erfolgte über 100 Epochen mit einer Batchgröße von 32 unter Berücksichtigung von Klassengewichten. Der Adam-Optimierer mit der Lernrate $\eta = 1e-4$ wurde für das Training eingesetzt, da er sich aufgrund seiner adaptiven Lernratenanpassung als besonders effizient für tiefere neuronale Netzwerke erwiesen hat (vgl. \citealt[S. 356-357]{Geron:2019}). Zur Optimierung des Trainingsprozesses wurde das Callback zur frühen Beendigung (\texttt{\colorbox{gray!20}{EarlyStopping}}) implementiert, sodass das Training abgebrochen wird, wenn der Validierungsverlust über 10 aufeinanderfolgende Epochen keine Verbesserung zeigte (vgl. \citet[S. 315-316]{Geron:2019}). Die Integration des Callbacks führte zu einer Vermeidung vom Overfitting sowie einer Verbesserung der Modellkonvergenz.

\subsection{Vortrainiertes CNN-Modell ResNet-50}
\subsubsection{Modellarchitektur}

ResNet von \citet[]{He:2016}, auch als Residual Network bezeichnet, konnte im Jahr 2015 die \ac{ilsvrc} \citet[]{Russakovsky:2015} für sich entscheiden und wies dabei eine bemerkenswert niedrige Fehlerrate von unter 3,6\% auf. ResNet ermöglicht den Aufbau extrem tiefer neuronaler Netzwerke mithilfe von Skip-Verbindungen. Diese Verbindungen bewirken, dass die Eingabe eines Layers unmittelbar dem Ausgang eines späteren Layers zugeführt wird. Infolgedessen modelliert das Netzwerk nicht die Zielfunktion direkt, sondern die Differenz zwischen der Zielfunktion und der Eingabe (Residual Learning). Diese Technik erleichtert den Trainingsprozess und reduziert die Probleme tiefer Netzwerke, wie das Verschwinden von Gradienten (engl. \textit{vanishing gradient}). Auch wenn einzelne Schichten während des Trainings suboptimal lernen, kann das Netzwerk durch die effiziente Signalweiterleitung dennoch Fortschritte erzielen. (Vgl. \citealt[S. 471]{Geron:2019}).
\\
Die Architektur von ResNet umfasst sogenannte Residual Units (RUs), die aus mehreren konvolutionalen Schichten mit Skip-Verbindungen bestehen. Jede RU enthält Batch-Normalisierung und ReLU-Aktivierungen, wobei in der Regel 3×3-Kernel verwendet werden, um räumliche Dimensionen der Merkmalskarten beizubehalten. Zur Reduktion der Höhe und Breite der Merkmalskarten kommt eine Stride-2-Konvolution zum Einsatz, während 1×1-Schichten die Dimensionen der Skip-Verbindungen anpassen. (Vgl. \citealt[S. 471-473]{Geron:2019}).
\\
Ein wesentlicher Bestandteil von ResNet-50 sind die sogenannten Bottleneck-Residual Units. Diese Units setzen sich aus aus drei konvolutionalen Schichten: einer 1×1-Schicht zur Reduktion der Merkmalskarten, einer 3×3-Schicht zur Mustererkennung und einer weiteren 1×1-Schicht zur Wiederherstellung der ursprünglichen Merkmalskartentiefe. Die 1×1-Schichten, vielfach auch als Bottleneck-Schichten (engl. \textit{bottleneck layer} bezeichnet, fokussieren auf Muster entlang der Tiefe, reduzieren die Dimensionalität und senken dadurch den Rechenaufwand sowie die Anzahl der Modellparameter. Dies verbessert die Effizienz und Generalisierungsfähigkeit des Modells. Die Abfolge aus 1×1-, 3×3- und 1×1-Schicht agiert wie eine besonders leistungsfähige, zusammengefasste konvolutionale Schicht. Im Vergleich zu einem einfachen linearen Klassifikator, der über ein Bild "`gescannt"' wird, ermöglicht diese Schichtkombination die Erkennung komplexerer Muster, indem ein kleines neuronales Netzwerk über das Bild verschoben wird. (Vgl. \citealt[S. 467-468; 474]{Geron:2019}).
\\
Das ResNet-50 ist in verschiedene Blöcke unterteilt, die jeweils eine steigende Anzahl an Merkmalskarten enthalten. Dabei handelt es sich um drei Residual Units mit 256, vier mit 512, sechs mit 1024 sowie drei mit 2048 Merkmalskarten. Die Implementierung von Bottleneck-Schichten resultiert in einer gesteigerten Effizienz der Verarbeitung, ohne dass dies eine Reduktion der Modellleistung bedingt. (Vgl. \citealt[S. 5]{He:2016}).\\
Abbildung \ref{fig:6.3} illustriert die Architektur von ResNet-50, einschließlich der Bottleneck-Schichten und der Residual Units.
\begin{figure}[ht!]
    \centering
        \includegraphics[height=1\textwidth]{BA/Abbildungen/ResNet Achitektur.png}
        \caption{ResNet-50 Modellarchitektur, entnommen aus \citet[S. 1020]{Bendjillali:2020}.}
    \label{fig:6.3}
\end{figure}

\subsubsection{Praktische Anwendung des Modells}

Die Wahl des ResNet-50 im Rahmen dieser Arbeit wurde aufgrund seiner tiefen, aber gleichzeitig effizienten Architektur getroffen, die durch Bottleneck-Schichten und Residual Learning auch bei komplexen Aufgaben hohe Leistungsfähigkeit zeigt. Die Verwendung eines vortrainierten Modells bringt zusätzliche Vorteile, darunter kürzere Trainingszeiten, reduzierte Parameteranzahl und die Möglichkeit, auf ImageNet extrahierte Merkmale für domänenspezifische Aufgaben zu nutzen.
\\
Im ersten Schritt wurde das ResNet-50-Modell dahingehend angepasst, dass ausschließlich die neue Ausgabeschicht trainiert wurde. Dies hatte zur Konsequenz, dass alle anderen Schichten eingefroren blieben und ihre vortrainierten Parameter behielten. Durch diese Methode blieben die vortrainierten Gewichte erhalten, während die Anpassung auf den spezifischen Datensatz fokussiert wurde. Die Implementierung der neuen Ausgabeschicht erfolgte durch eine Kombination verschiedener Schritte. In einem ersten Schritt wurde die Ausgabe vom ResNet-50 in eine Global-Average-Pooling-Schicht überführt, wodurch eine Reduktion der Dimensionen sowie eine robuste Repräsentation der Merkmale erzielt wurde. Im Anschluss wurde eine dichte Schicht mit 1024 Neuronen und ReLU-Aktivierung implementiert, um Overfitting zu reduzieren. Zu diesem Zweck wurde eine Dropout-Schicht (50\%) hinzugefügt. Die Ausgabeschicht besteht aus vier Neuronen und einer Softmax-Aktivierung. \\
Das Modell wurde mit dem Adam-Optimierer konfiguriert. Eine Lernrate $\eta$ von $1e-4$ wurde gewählt, basierend auf stabilen und schnellen Ergebnissen in vorangegangenen Tests. In Bezug auf die Verlustfunktion wurde die Wahl auf \ac{cce} getroffen, da es sich um ein mehrklassiges Klassifizierungsproblem handelt. Die Überwachung der Modellleistung erfolgte anhand der Accuracy als Metrik, sowohl während des Trainings als auch auf den Validierungsdaten.
\\
Das Training wurde mit 100 Epochen durchgeführt, wobei das Callback (\texttt{\colorbox{gray!20}{EarlyStopping}}) integriert wurde. Des Weiteren wurden Klassengewichte eingeführt, um die Auswirkungen der ungleichen Klassenverteilung zu mildern und das Modell für seltener vertretene Klassen zu sensibilisieren.
\\
Nach Abschluss des ersten Trainingsschritts wurde die Ausgabeschicht des Modells beibehalten und die letzten 20 Schichten zum \ac{ft} freigegeben. Diese Schichten enthalten die komplexesten Merkmalsextraktionen und wurden durch das \ac{ft} an die Merkmale des spezifischen Datensatzes angepasst. Um ein Overfitting zu vermeiden, wurde die Lernrate $\eta$ auf $1e-5$ reduziert. Dadurch ist das Modell in der Lage, feinere Anpassungen vorzunehmen, ohne die vortrainierten Gewichte signifikant zu verändern.

\subsubsection{\acf{hpo}}

Die Auswahl geeigneter \ac{hp} ist ein entscheidender Schritt in der Optimierung von \ac{dnn}, da sie maßgeblich die Lernfähigkeit und Generalisierung beeinflusst. Im Rahmen der \ac{hpo} wurde der Algorithmus \acf{gs} angewendet. \ac{gs} stellt eine systematische Methode zur Optimierung von \ac{hp} dar, bei der für jeden \ac{hp} eine endliche Menge potenzieller Werte definiert wird. Im Anschluss erfolgt die Modellierung aller Kombinationen der zuvor definierten Werte, welche einer Evaluierung unterzogen werden. Das Ziel besteht in der Identifikation derjenigen \ac{hp}-Kombination, welche den geringsten Validierungsfehler bzw. die beste Validierungsgenauigkeit aufweist. Die Methode basiert auf der Erstellung eines kartesischen Produkts aus den Werten der \ac{hp} und umfasst für jede Kombination ein Training. (Vgl. \citealt[S. 432]{Goodfellow:2017}).
\\
Der Prozess von \ac{gs} kann in drei wesentliche Schritte unterteilt werden: Zunächst werden die Wertebereiche definiert. Für numerische \ac{hp} wie die Lernrate wird häufig eine logarithmische Skalierung verwendet, während diskrete Parameter durch eine begrenzte Anzahl spezifischer Werte definiert werden. Anschließend werden alle möglichen Kombinationen im Rahmen einer systematischen Evaluierung getestet. Dabei steigt die Trainingszeit exponentiell mit der Anzahl der \ac{hp} nach $O(n^m)$, wobei $n$ die Anzahl der Werte pro \ac{hp} und $m$ die Anzahl der \ac{hp}. Zum Beispiel würde die Optimierung von drei \ac{hp} mit jeweils fünf möglichen Werten zu $5^3 = 125$ Kombinationen führen. (Vgl. \citealt[S. 434]{Goodfellow:2017}).
\\
\ac{gs} eignet sich besonders für Probleme mit wenigen \ac{hp}, da das Verfahren trotz exponentiell wachsender Rechenzeit eine vollständige Exploration des Parameterraums erlaubt. Aufgrund dieser Einfachheit und Gründlichkeit ist \ac{gs} trotz seiner Einschränkungen bei komplexen Parameterräumen in der Praxis weit verbreitet. (Vgl. \citealt[S. 434]{Goodfellow:2017}).
\\
Im Rahmen dieser Arbeit wurde \ac{gs} eingesetzt, um die Anzahl der trainierbaren Schichten (\texttt{\colorbox{gray!20}{trainable\_layers}}) im Modell ResNet-50 zu optimieren. Dieser \ac{hp}s wurde auf Basis früherer Trainingsdurchläufe gewählt, bei denen das Training der letzten 20 Schichten vielversprechende Ergebnisse lieferte. Die Optimierung erfolgte durch Variation der trainierbaren Schichten im Bereich von 1 bis 25, um den Einfluss dieser Schichten auf die Modellleistung systematisch zu untersuchen.
\\
\ac{gs} erwies sich als geeignet, da lediglich ein einziger \ac{hp} optimiert werden musste, was den Rechenaufwand trotz exponentieller Skalierung überschaubar hielt. Des Weiteren ermöglichte der Algorithmus im Gegensatz zum Algorithmus \acf{rs}, der die Parameterkombinationen zufällig auswählt, eine vollständige Abdeckung des definierten Bereichs, sodass eine präzise Analyse der Auswirkungen auf die Fähigkeit des Modells, domänenspezifische Merkmale zu erlernen, möglich war (vgl. \citealt[S. 433]{Goodfellow:2017}).
\\
Das Modell ResNet-50 wurde in seiner ursprünglichen Architektur mit der neuen Ausgabeschicht geladen. Zu Beginn des Prozesses wurden sämtliche Schichten eingefroren, sodass keine Modifikationen an ihnen vorgenommen wurden. Im Anschluss wurden lediglich die letzten Schichten entsprechend der zuvor definierten Anzahl für \ac{ft} freigegeben, um die Optimierung auf domänenspezifische Merkmale zu konzentrieren.
\\
Für die Kompilierung des Modells wurde der Adam-Optimierer mit einer Lernrate von $\eta = 1e-5$ verwendet. Das Training erfolgte auf den Trainings- und Validierungsdatensätzen, wobei die Leistungsfähigkeit des Modells kontinuierlich evaluiert wurde. Zur Durchführung der GS wurde ein Tuner-Objekt erstellt, welches die Modellarchitektur sowie die Optimierungsziele referenzierte. Die Anzahl der Versuche wurde auf maximal 25 konfiguriert, wobei jeder Versuch unterschiedliche Werte für die trainierbaren Schichten testete. Um die Stabilität der Ergebnisse zu gewährleisten, wurde jede Konfiguration dreimal wiederholt. Nach Abschluss der \ac{gs} wurde die Konfiguration mit dem niedrigsten Validierungsverlust als optimal identifiziert. Die resultierenden \ac{hp} wurden direkt aus dem Tuner abgerufen, um die optimale Anzahl trainierbarer Schichten zu bestimmen und die Leistung des Modells ResNet-50 weiter zu optimieren.
\\
Neben der manuellen Optimierung und der \ac{gs} stellt die Bayessche Optimierung (BO) eine weitere effiziente Alternative dar. Dieser Ansatz modelliert die Funktion der Hyperparameterleistung auf Basis einer Wahrscheinlichkeitsverteilung und optimiert iterativ, was insbesondere bei komplexen Modellen mit hohen Trainingskosten von Vorteil ist. Des Weiteren erlaubt die Methode die gezielte Exploration von Bereichen des Hyperparameterraums, wodurch sich eine Beschleunigung der Optimierung erzielen lässt. (Vgl. \citealt[S. 9-12]{Bischl:2023}).
\newpage

\section{Performancevergleich der Modelle}

\subsection{Darstellung der Ergebnisse}

Das selbst entwickelte \ac{cnn}-Modell erreichte auf dem Testdatensatz eine Testgenauigkeit von 98,10\% und einen Testverlust von 0,1027. Die in Abbildung \ref{fig:7.1} dargestellten Training- und Validierungslernkurven verdeutlichen eine nahezu kontinuierliche Verbesserung sowohl der Genauigkeit als auch des Verlusts, wobei einzelne starke Schwankungen zu beobachten sind. 

\begin{figure}[ht]
    \centering
    \includegraphics[width=1\textwidth]{BA/Abbildungen/Selbsterstelltes CNN _ Acc_Loss.png}
    \caption{Lernkurven des selbst entwickelten \ac{cnn}s: Links die Genauigkeit und rechts der Verlust.}
    \label{fig:7.1}
\end{figure}

Die in Abbildung \ref{fig:7.2} dargestellte \ac{cm} zeigt eine hohe Precision sowie einen exzellenten Recall für die Klasse notumor. Demgegenüber weisen die F1-Scores der Klassen glioma und meningioma eine leichte Unterperformance gegenüber den Werten der übrigen Klassen auf, wie aus dem \ac{cr} in Abbildung \ref{fig:7.3} ersichtlich ist.

\begin{figure}[h!]
    \centering
    % Erste Abbildung
    \begin{minipage}[t]{0.51\textwidth}
        \centering
        \includegraphics[width=\textwidth]{BA/Abbildungen/Selbsterstelltes CNN _ CM.png}
        \caption{\ac{cm} vom selbst entwickelten \ac{cnn}.}
        \label{fig:7.2}
    \end{minipage}
    \hfill
    % Zweite Abbildung
    \begin{minipage}[t]{0.45\textwidth}
        \centering
        \includegraphics[width=\textwidth]{BA/Abbildungen/Selbsterstelltes CNN _ CR.png}
        \caption{\ac{cr} vom selbst entwickelten \ac{cnn}.}
        \label{fig:7.3}
    \end{minipage}
\end{figure}

Das Modell ResNet-50, welches lediglich um eine an den Datensatz angepasste Ausgabeschicht erweitert und trainiert wurde, erzielte eine Testgenauigkeit von 76,45\% bei einem Verlust von 0,5825. Die Analyse der Lernkurven, dargestellt in Abbildung \ref{fig:7.4}, deutet auf eine limitierte Generalisierungsfähigkeit hin.

\begin{figure}[ht]
    \centering
    \includegraphics[width=1\textwidth]{BA/Abbildungen/ResNet Output Acc_Loss.png}
    \caption{Lernkurven des ResNet-50 mit angepasster Ausgabeschicht: Links die Genauigkeit, rechts der Verlust.}
    \label{fig:7.4}
\end{figure}

Wie aus der \ac{cm} (vgl. Abbildung \ref{fig:7.5}) ersichtlich ist, zeigte das Modell Schwierigkeiten bei der Klassifikation der Klassen glioma und meningioma. Demgegenüber konnten die Klassen notumor und pituitary mit einer höheren Anzahl korrekter Klassifikationen zugeordnet werden. Dies wird durch den \ac{cr} (vgl. Abbildung \ref{fig:7.6}) bestätigt, der einen Macro Average F1-Score von 0,75 ausweist. Dies unterstreicht, dass eine einfache Anpassung der Ausgabeschicht ohne ein weitergehendes \ac{ft} nicht ausreichend ist.

\begin{figure}[h!]
    \centering
    % Erste Abbildung
    \begin{minipage}[t]{0.52\textwidth}
        \centering
        \includegraphics[width=\textwidth]{BA/Abbildungen/ResNet Output CM.png}
        \caption{\ac{cm} vom ResNet-50 mit angepasster Ausgabeschicht.}
        \label{fig:7.5}
    \end{minipage}
    % Zweite Abbildung
    \begin{minipage}[t]{0.45\textwidth}
        \centering
        \includegraphics[width=\textwidth]{BA/Abbildungen/ResNet Output CR.png}
        \caption{\ac{cr} vom ResNet-50 mit angepasster Ausgabeschicht.}
        \label{fig:7.6}
    \end{minipage}
\end{figure}

Das \ac{ft} der letzten 20 Schichten des ResNet-50 führte zu einer signifikanten Steigerung der Ergebnisse. Die Evaluierung des Testdatensatzes resultierte in einer Genauigkeit von 97,43\% sowie einem Verlust von 0,1051. Die Lernkurven (vgl. Abbildung \ref{fig:7.7}) demonstrieren eine signifikante Steigerung der Generalisierungsfähigkeit im Vergleich zum vorherigen Ansatz. 

\begin{figure}[ht]
    \centering
    \includegraphics[width=1\textwidth]{BA/Abbildungen/ResNet 20 Acc_Loss.png}
    \caption{Lernkurven des ResNet-50 mit \ac{ft}: Links die Genauigkeit, rechts der Verlust.}
    \label{fig:7.7}
\end{figure}

Die \ac{cm} (Abbildung \ref{fig:7.8}) zeigt, dass das \ac{ft} eine präzisere Differenzierung der Klassen glioma und meningioma ermöglicht hat. Ergänzend verdeutlicht der \ac{cr} (Abbildung \ref{fig:7.9}) die gesteigerte Leistungsfähigkeit des Modells, was sich in einem Macro Average F1-Score von 0,97 für alle Klassen widerspiegelt.

\begin{figure}[h!]
    \centering
    % Erste Abbildung
    \begin{minipage}[t]{0.52\textwidth}
        \centering
        \includegraphics[width=\textwidth]{BA/Abbildungen/ResNet 20 CM.png}
        \caption{\ac{cm} vom ResNet-50 mit \ac{ft}.}
        \label{fig:7.8}
    \end{minipage}
    % Zweite Abbildung
    \begin{minipage}[t]{0.45\textwidth}
        \centering
        \includegraphics[width=\textwidth]{BA/Abbildungen/ResNet 20 CR.png}
        \caption{\ac{cr} vom ResNet-50 mit \ac{ft}.}
        \label{fig:7.9}
    \end{minipage}
\end{figure}

Die \ac{hpo} vom Resnet-50 mit der Anzahl der trainierbaren Schichten als \ac{hp} demonstrierte, dass sowohl die Genauigkeit als auch der Verlust ab einer Anzahl von 19 Schichten eine Stabilisierung erfahren und somit ähnliche Ergebnisse im Durchschnitt erzielt werden konnten. Der optimale \ac{hp} resultierte in einer Anzahl von 23 Schichten mit einer Testgenauigkeit von 97,66\% bei einem Testverlust von 0,1097. Die Abbildung \ref{fig:7.10} veranschaulicht den Effekt des \ac{hp}s auf die Genauigkeit und den Verlust. Die \ac{cm} und der \ac{cr} mit dem besten \ac{hp} sind in den Abbildungen \ref{fig:7.11} und \ref{fig:7.12} dargestellt. 
\\
Die Ergebnisse nach \ac{hpo} (23 Schichten) belegen die Konsistenz der Performance des Modells. Die erzielte Genauigkeit, der Verlust und die F1-Scores weichen nur geringfügig von den Resultaten mit 20 Schichten ab, was auf die Robustheit der Optimierung hindeutet.

\begin{figure}[ht]
    \centering
    \includegraphics[width=1\textwidth]{BA/Abbildungen/ResNet HPO Acc_Loss.png}
    \caption{Durchschnittliche Genauigkeit (links) und durchschnittlicher Verlust (rechts) in Abhängigkeit von der Anzahl der trainierbaren Schichten.}
    \label{fig:7.10}
\end{figure}

\begin{figure}[h!]
    \centering
    % Erste Abbildung
    \begin{minipage}[t]{0.52\textwidth}
        \centering
        \includegraphics[width=\textwidth]{BA/Abbildungen/ResNet HPO CM.png}
        \caption{\ac{cm} vom ResNet-50 \\ nach \ac{hpo}.}
        \label{fig:7.11}
    \end{minipage}
    % Zweite Abbildung
    \begin{minipage}[t]{0.45\textwidth}
        \centering
        \includegraphics[width=\textwidth]{BA/Abbildungen/ResNet HPO CR.png}
        \caption{\ac{cr} vom ResNet-50 nach \ac{hpo}.}
        \label{fig:7.12}
    \end{minipage}
\end{figure}

\subsection{Diskussion der Ergebnisse}

Die Analyse der Resultate beider Modelle offenbart eine beeindruckende Klassifikationsleistung von \ac{ht}. Das selbst entwickelte \ac{cnn} erreichte eine Genauigkeit von 98,1\%, während das feinjustierte ResNet-50-Modell eine Genauigkeit von 97,43\% erzielte. Die Ähnlichkeit der Resultate veranschaulicht die jeweiligen Stärken und Schwächen der Ansätze, die für die praktische und theoretische Anwendung von \ac{dl} in der medizinischen Bildanalyse relevant sind.
\\
Das \ac{cnn} profitierte von spezifischen Anpassungen an den verwendeten \ac{ds} und demonstrierte eine hohe Effizienz, insbesondere durch den Einsatz von Techniken wie Dropout, Batchnormalisierung und L2-Regularisierung. Im Rahmen des Modelltrainings wurden diverse Dropout-Raten (0,3, 0,4 und 0,5) evaluiert, wobei sich eine Rate von 0,5 als optimal erwies. Diese Rate gewährleistet das Gleichgewicht zwischen Regularisierung und Modellkapazität. Des Weiteren wurden diverse Konfigurationen der dichten Schicht mit 256, 512 und 1024 Neuronen evaluiert. Hierbei zeigte sich, dass eine Konfiguration mit 1024 Neuronen die beste Leistung hinsichtlich des F1-Scores und der Generalisierungsfähigkeit erzielte. Allerdings wies das Modell Anzeichen von Overfitting auf, insbesondere bei den Klassen glioma und meningioma. Dies lässt auf eine begrenzte Datenvielfalt in diesen Kategorien schließen. Dies verdeutlicht die Notwendigkeit einer besseren Datenrepräsentation, um die Modellleistung zu stabilisieren. Gleichzeitig wies das \ac{cnn} eine hohe Genauigkeit auf, ohne dass hierfür ein umfassendes \ac{tl} oder vortrainierte Gewichte erforderlich waren. Dies unterstreicht die Wirksamkeit der gewählten Modellarchitektur und Regularisierungstechniken.
\\
Im Gegensatz dazu wies das ResNet-50-Modell eine höhere Robustheit gegenüber Overfitting auf, was auf die Lernkurven mit wenigen starken Schwankungen zu sehen ist. Dies lässt sich auf die Vorteile von \ac{tl} und die bereits in den vortrainierten Schichten enthaltenen generalisierten Fähigkeiten zur Merkmalsextraktion zurückführen. Die Resultate demonstrieren jedoch auch, dass eine bloße Anpassung der Ausgabeschicht ohne weiterführendes \ac{ft} nicht hinreichte, um optimale Resultate zu erzielen. Erst nach einem gezielten \ac{ft} von Schichten konnte eine vergleichbare Performance wie beim selbst entwickelten \ac{cnn} erreicht werden. Dies lässt den Schluss zu, dass die Anzahl der trainierbaren Schichten eine zentrale Rolle beim \ac{ft} solcher Modelle spielt.
\\
Der mit dem ResNet-50-Modell erzielte Genauigkeitswert von 97,66\% entspricht in etwa den in der Literatur berichteten Ergebnissen. \citet{Ismael:2020} nutzten ResNet-50, ergänzt durch Datenaugmentation, und erzielten einen Genauigkeitswert von 97\%. \citet{Tandel:2021} verwendeten einen mehrheitswahlbasierten Ensemble-Ansatz und kamen auf ähnlich hohe Werte von 96,54\%. Im Gegensatz dazu erzielte das in dieser Arbeit präsentierte ResNet-50-Modell eine vergleichbare Leistung, ohne auf Datenaugmentation zurückzugreifen.
\\
Beide Modelle wiesen eine nahezu perfekte Klassifikation der Klassen notumor und pituitary auf (F1-Scores zw. 0,99 und 1), was darauf hindeutet, dass die morphologischen Merkmale dieser Klassen im Datensatz gut repräsentiert sind und somit von den Modellen effektiv erlernt werden konnten. Diese Ergebnisse unterstreichen die Bedeutung der Datenrepräsentativität bei der Modellentwicklung und -evaluierung.
\\
Die Gegenüberstellung beider Ansätze verdeutlicht zudem den Zielkonflikt zwischen Modellspezialisierung und Generalisierbarkeit. Das selbst entwickelte \ac{cnn} erzielte eine hohe Leistung auf dem gegebenen Datensatz, wobei spezifische Anpassungen vorgenommen wurden. Das ResNet-50-Modell hingegen zeichnet sich durch seine vortrainierten Schichten aus, wodurch eine größere Flexibilität und Robustheit gewährleistet wird. Die Resultate liefern wertvolle Erkenntnisse hinsichtlich der praktischen Anforderungen und der Einsatzmöglichkeiten von \ac{dl}-Modellen in der medizinischen Diagnostik. In einem klinischen Kontext könnte die Implementierung solcher Modelle die diagnostische Effizienz steigern, indem sie Radiologen bei der Differenzierung von Tumortypen unterstützen und die Arbeitsbelastung verringern. Dennoch ist eine Validierung mit unabhängigen klinischen Datensätzen erforderlich, um die Robustheit und Generalisierbarkeit der Modelle zu gewährleisten.
\\
Die Ergebnisse zeigen, dass die Wahl zwischen einem speziell entwickelten Modell und einem vortrainierten Ansatz von den spezifischen Anforderungen sowie den verfügbaren Ressourcen abhängig ist. Beide Ansätze besitzen ihre Relevanz und können durch gezielte Optimierungen eine weitere Verbesserung erfahren. Dies dürfte künftige Forschungsarbeiten zu Modellarchitekturen und Datenverarbeitung anregen.
\newpage

\section{Zukunftsaussichten}

Die Weiterentwicklung der Klassifizierung von \ac{ht} mithilfe von \ac{dl}-Algorithmen eröffnet zahlreiche Perspektiven zur Leistungssteigerung und Bewältigung bestehender Herausforderungen. Ein wesentliches Problem stellt das Overfitting dar, welches durch eine Erweiterung des Datensatzes reduziert werden könnte. Eine Lösung könnte in der Kombination verschiedener öffentlich zugänglicher \ac{mrt}-Datensätze oder der Ergänzung um \ac{ct}-Bilder bestehen. Die üblicherweise begrenzte Größe medizinischer Bilddatensätze sowie deren hohe Spezialisierung könnten durch einen umfassenderen Datensatz kompensiert werden. Dies würde die Robustheit und Generalisierungsfähigkeit der Modelle fördern und zu stabileren Trainingsergebnissen sowie einer besseren Modellvalidität und -anpassungsfähigkeit gegenüber neuen, unbekannten Bildmustern führen. \\
Ein weiterer wesentlicher Ansatzpunkt zur Verbesserung der Modellleistung stellt die Optimierung der \ac{hp} dar. Ein fortschrittliches Verfahren in diesem Bereich ist das Iterated Racing von \citet{Birattari:2010}, welches von \citet[]{Bischl:2023} als eine der führenden Methoden der \ac{hpo} hervorgehoben wird. Das Verfahren fokussiert die Rechenleistung auf vielversprechende Konfigurationen, indem weniger geeignete Parameterkombinationen bereits in frühen Phasen eliminiert werden. Dies erlaubt eine effizientere und präzisere Suche nach optimalen Parametereinstellungen bei gleichzeitiger Schonung von Rechenressourcen. Diese Methode kann dazu beitragen, künftige Optimierungsprozesse zu beschleunigen und die Genauigkeit der Klassifikationsmodelle weiter zu steigern. \\
Eine weiterer innovativer Ansatz zur Steigerung der Leistungsfähigkeit stellt der Einsatz vortrainierter Modelle dar, welche auf verschiedene Tumorarten spezialisiert sind. Eine derartige Vortrainierung ermöglicht es dem Modell, allgemeine Tumorcharakteristika besser zu erkennen, wodurch sich zugleich seine Fähigkeit erhöht, spezifische \ac{ht}-Merkmale zu klassifizieren. Auch die Implementierung eines Segmentierungsansatzes birgt ein vielversprechendes Potenzial. Hierbei könnte das Modell gezielt lernen, Tumorstrukturen in Bildern zu identifizieren und anschließend zu klassifizieren. Der zusätzliche Vorverarbeitungsschritt könnte zu differenzierteren Einblicken in die Tumorbiologie führen, was insbesondere für die präzisere Erkennung seltener oder atypischer \ac{ht} von Bedeutung wäre. \\
Im Rahmen künftiger Forschungsarbeiten wäre es vorteilhaft, die Eigenschaften unterschiedlicher Modellarchitekturen systematisch zu analysieren und deren Leistungsfähigkeit im Hinblick auf medizinische Bilddaten zu evaluieren. Eine detaillierte Bewertung der Merkmalsextraktionsfähigkeiten könnte dazu beitragen, die am besten geeignete Architektur für die Klassifikation von \ac{ht} zu identifizieren und somit die Leistungsfähigkeit weiter zu steigern. Derartige Vergleiche könnten zudem Aufschluss über spezifische Architekturanpassungen geben, die sich besonders für medizinische Bildanalysen eignen.
\\
Ein vielversprechender Forschungsbereich ist zudem die Erweiterung der Modelle auf 3D-\ac{mrt}-Daten, um volumetrische Informationen besser auszunutzen. Dies könnte durch die Integration von 3D-\ac{cnn}s oder hybridisierten Modellen erfolgen, welche sowohl 2D- als auch 3D-Daten kombinieren. Darüber hinaus besteht ein Bedarf an der Entwicklung von Modellen, welche interpretierbare und klinisch relevante Ergebnisse liefern, um die Akzeptanz durch Mediziner zu fördern.
\\
Die Umsetzung der dargestellten Optimierungsansätze konnte im Rahmen dieser Arbeit jedoch nicht realisiert werden, da sie einen signifikanten Aufwand an Zeit und Rechenkapazität erfordert. Umfassende Datenerweiterungen, fortgeschrittene Transfer- oder Segmentierungsansätze sowie eine Hyperparameteroptimierung mittels Iterated Racing würden eine erhebliche Steigerung der verfügbaren Rechenleistung sowie einen erweiterten Zeitrahmen verlangen. Zukünftige Forschungsprojekte könnten daher gezielt auf diesen Ansätzen aufbauen und damit das Potenzial moderner \ac{dl}-Methoden zur \ac{ht}-Klassifikation weiter ausschöpfen.
\newpage

\section{Schlussfolgerung}

Im Rahmen dieser Arbeit wurde die Klassifikation von \ac{ht} unter Zuhilfenahme von \ac{dl}-Algorithmen untersucht. Zu diesem Zweck wurde ein eigener \ac{cnn}-Ansatz mit 7.023 medizinischen Bildern trainiert. Die Herausforderung, ohne Datenaugmentation zu arbeiten, wurde durch die realistische Nähe zu klinischen Daten motiviert, was insbesondere im Hinblick auf das Risiko des Overfitting von Bedeutung war. Durch den Einsatz von Regularisierungstechniken wie Dropout, Batch-Normalisierung und L2-Regularisierung konnte eine Reduktion des Overfittings erzielt werden. Obwohl der Datensatz lediglich 7.023 Bilder umfasst, konnte das Modell ein Testgenauigkeit von 98,1\% sowie einen Testverlust von 0,1027 erzielen, was als gut zu bewerten ist.
\\
Zudem wurde ein Vergleich des eigenentwickelten Modells mit dem vortrainierten ResNet-50-Modell vorgenommen, um die Vorzüge von \ac{tl} und von \ac{ft} zu nutzen. Die ursprüngliche Architektur des ResNet-50 wurde beibehalten, wobei die Ausgabeschicht angepasst wurde. Das Modell wurde mit einer definierten Anzahl an trainierbaren Schichten an dem Datensatz feingetunt. Die Durchführung von \ac{hpo} ermöglichte die Untersuchung des Einflusses der Anzahl trainierbarer Schichten auf die Modellleistung. Die Ergebnisse zeigten, dass die Konsistenz der Resultate ab einer Anzahl von 19 Schichten gegeben war. Das optimierte vortrainierte Modell erreichte eine Testgenauigkeit von 97,66\% bei einem Testverlust von 0,1097. 
\\
Die ursprüngliche Hypothese, dass das vortrainierte Modell zu besseren Resultaten führen würde, konnte in dieser Untersuchung nicht bestätigt werden. Das vortrainierte Modell erzielte ähnliche Resultate zum selbst entwickelten Modell, wies jedoch kein Overfitting auf, wodurch es sich unter bestimmten Voraussetzungen für die Klassifikation von \ac{ht} als potenziell besser geeignet erwies. Für Experten im Bereich des \ac{ml}s und des \ac{dl}s könnte das selbst entwickelte Modell aufgrund der spezifischen Anpassung an den Datensatz sowie der Möglichkeit einer erweiterten \ac{hpo} von besonderem Vorteil sein. In Bezug auf die Anwendbarkeit in klinischen Umgebungen, insbesondere zur Unterstützung von Radiologen, stellt das vortrainierte Modell ResNet-50 aufgrund seiner einfachen Implementierung und der guten Ergebnisse eine attraktive Wahl dar.
\\
In künftigen Forschungsarbeiten sollte untersucht werden, inwiefern eine Erweiterung des Datensatzes durch synthetische Daten oder zusätzliche klinische Datensätze eine weitere Verbesserung der Performance beider Modelle ermöglicht. Des Weiteren erscheint eine Vertiefung der \ac{hpo} sowie eine Prüfung des Einflusses zusätzlicher Regularisierungstechniken sinnvoll, um das Modell weiter zu optimieren und eine Robustheit gegen Overfitting zu erzielen. 
\\
Zusammenfassend lässt sich sagen, dass \ac{dl} ein vielversprechendes Werkzeug zur Unterstützung der medizinischen Diagnostik darstellt, dessen Potenzial durch kontinuierliche Forschung und Optimierung noch weiter ausgeschöpft werden kann.
\newpage


%%%%%%%%%%%%%%%%%%%%%%%%%%%%%%%%%%%%%%%%%%%%%%%%%%%%%%%%%%%%%%%%%%%%%%%
%%%%%%%%%%%%%%%%%%%%%%%%%%%%%%%%%%%%%%%%%%%%%%%%%%%%%%%%%%%%%%%%%%%%%%%
%%%%%%%%%%%%%%%%%%%%%%%%%%%%%%%%%%%%%%%%%%%%%%%%%%%%%%%%%%%%%%%%%%%%%%%

%\pagenumbering{Roman}
\bibliographystyle{autorjahrdidiDE}  
% Dies ist der verwendete Stil für Zitation und Literaturverzeichnis
% Die Datei autorjahrdidiDE.bst muss in das aktuelle Arbeitsverzeichnis kopiert werden
\bibliography{Literatur}	           
% Literaturverzeichnis liegt in der Datei Literatur

%\newpage
%\section*{Anhang}
%\addcontentsline{toc}{section}{Anhang}
\end{document}											 
% Ende des Dokuments